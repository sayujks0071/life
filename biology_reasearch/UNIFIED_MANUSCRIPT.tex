\begin{abstract}
Living systems routinely maintain structure against gravity, from plant stems that grow upward to vertebrate spines that adopt robust S-shaped profiles. We develop a quantitative framework that interprets this behavior as \emph{biological countercurvature}: information-driven modification of the effective geometry experienced by a body in a gravitational field. An information--elasticity coupling (IEC) model of spinal patterning is combined with three-dimensional Cosserat rod mechanics (PyElastica~\cite{pyelastica_zenodo,gazzola2018forward}), treating the rod in gravity as an analog spacetime and the IEC information field $I(s)$ as a source of effective countercurvature.

Along the body axis $s$, we define a biological metric $d\ell_{\mathrm{eff}}^{2} = g_{\mathrm{eff}}(s)\,ds^{2}$, where the conformal factor $g_{\mathrm{eff}}(s)$ depends on the local amplitude and gradient of $I(s)$. Using this countercurvature metric, we introduce a normalized geodesic curvature deviation $\widehat{D}_{\mathrm{geo}}$ that measures how far information-shaped equilibrium curvature profiles depart from gravity-selected profiles. Across canonical simulations---human-like spinal S-curves, plant-like stems, and microgravity adaptation---$\widehat{D}_{\mathrm{geo}}$ separates gravity-dominated, cooperative, and information-dominated regimes in $(\chi_{\kappa},g)$ space, where $\chi_{\kappa}$ controls information-to-curvature coupling and $g$ denotes gravitational strength.

To probe pathology within the same framework, we introduce a small, localized thoracic asymmetry in the information field or lateral rest curvature and track coronal-plane deformations. In the gravity-dominated regime this perturbation yields negligible lateral deviation. In the information-dominated corner of the phase diagram, the same small asymmetry is amplified into a scoliosis-like symmetry-broken branch with clinical-scale Cobb angles, providing a physical explanation for the window of vulnerability during adolescent growth. Our findings suggest that biological countercurvature is a fundamental principle of morphogenesis, offering a unified geometric perspective on how genetic information maintains structural integrity against environmental forces across diverse organismal forms and scales.
\end{abstract}
\section{Introduction}

\subsection{The puzzle of spinal curvature under gravity}
Living systems do not simply obey gravity; they negotiate with it. While a passive elastic beam clamped at one end and subject to gravity will sag into a monotonic C-shape, biological structures such as plant stems and vertebrate spines adopt complex, robust geometries that defy this passive tendency. The human spine, in particular, maintains a characteristic S-shaped sagittal profile (cervical and lumbar lordosis, thoracic kyphosis) that is critical for bipedal posture and shock absorption~\cite{white_panjabi_spine}. This shape is not merely a reaction to load but an intrinsic, actively maintained ``counter-curvature.''

\subsection{Developmental genetic patterning}
The blueprint for this geometry is laid down during embryogenesis. The paraxial mesoderm segments into somites, driven by the segmentation clock and oscillating gene expression (e.g., Notch, Wnt, FGF)~\cite{pourquie2011vertebrate}. These segments acquire distinct identities through the expression of HOX and PAX genes, which specify the morphological characteristics of the resulting vertebrae~\cite{wellik2007hox}. However, the mechanism by which these discrete genetic codes are translated into the continuous, mesoscale geometry of the adult spine remains a fundamental open question.

\subsection{Prior biomechanical models and their limitations}
Classical spinal biomechanics treats the spine as a passive elastic structure: either a continuous Euler-Bernoulli beam or a multi-body linkage of rigid vertebrae connected by compliant discs~\cite{white_panjabi_spine}. While these models successfully capture static load distribution and predict failure modes under extreme loading, they cannot explain several key observations:
\begin{enumerate}[label=(\roman*)]
\item \textbf{Robustness across environments}: The characteristic S-curve is maintained across a wide range of gravitational loads, from parabolic flight (transient microgravity) to long-duration spaceflight~\cite{green2018spinal}.
\item \textbf{Developmental stability}: The spinal curvature emerges reliably during development despite variations in mechanical loading (e.g., differences in prenatal activity or postnatal weight-bearing).
\item \textbf{Idiopathic pathology}: Adolescent idiopathic scoliosis (AIS) arises without detectable structural asymmetries in vertebrae or discs~\cite{weinstein2008adolescent}, suggesting a control or patterning defect rather than a purely mechanical failure.
\end{enumerate}
Our IEC framework addresses these gaps by incorporating developmental information as an \emph{active geometric source} that modifies the effective metric experienced by the spine, rather than treating curvature as a passive reaction to external loads.

\subsection{Hypothesis: Information--Elasticity Coupling and effective metric}
We propose that developmental information acts as a field that modifies the effective geometry experienced by the spine. Drawing an analogy to General Relativity, where matter curves spacetime, we suggest that biological information curves the ``material manifold'' of the spine. We formalize this as an \emph{Information--Elasticity Coupling (IEC)} framework, where a genetic information field $I(s)$ modifies the rest curvature, stiffness, and active moments of the structure. In this view, the spine does not ``fight'' gravity; it settles into the geodesic of a curved biological metric $d\ell_{\mathrm{eff}}^2$ shaped by information.

\subsection{Contribution and overview}
In this work, we:
(i) Define the IEC model and a phenomenological ``biological metric'' that maps genetic information to geometric distortions.
(ii) Implement this framework in a 3D Cosserat rod simulation (using \texttt{PyElastica}) to model the spine under gravitational loading.
(iii) Demonstrate that the interplay between gravity and information selects specific ``spinal modes''---shifting the ground state from a passive C-shape to an active S-shape.
(iv) Show that in information-dominated regimes, this same mechanism can amplify small asymmetries into pathological, scoliosis-like deformities.
\section{Theory}

We propose that the robust S-shaped geometry of the spine arises not from passive mechanical equilibrium under gravity, but from an active \emph{counter-curvature} mechanism driven by developmental information. We formalize this using an Information--Elasticity Coupling (IEC) framework, where a scalar information field $I(s)$ modifies the effective geometry and energetics of a Cosserat rod.

\subsection{Geometry and parameterization}

Consider a slender rod parameterized by arc-length $s \in [0, L]$. The configuration is defined by a centerline curve $\mathbf{r}(s) \in \mathbb{R}^3$ and a director frame $\{\mathbf{d}_1, \mathbf{d}_2, \mathbf{d}_3\}(s)$ describing the orientation of cross-sections. The rod deforms under a gravitational field $\mathbf{g} = -g \hat{\mathbf{e}}_z$. In the absence of biological regulation, such a rod would sag into a C-shape (kyphosis) or buckle.

\subsection{Information field from developmental patterning}

We introduce a scalar field $I(s)$ representing the spatial distribution of developmental identity along the axis (e.g., HOX gene expression domains or segmentation clock outputs). This field acts as a ``morphogenetic coordinate,'' encoding the target geometry. For the spine, $I(s)$ is modeled as a bimodal distribution peaking in the cervical and lumbar regions, corresponding to the lordotic curves required for upright posture.

\subsection{The Information--Cosserat Manifold}

The Human spine is modeled as a one-dimensional Cosserat rod embedded in three-dimensional Euclidean space. However, in the IEC framework, we treat the rod's reference configuration not as a flat segment, but as a manifold whose intrinsic geometry is warped by a developmental information field $I(s)$. We draw a formal analogy to General Relativity (GR), where the geometry of spacetime $g_{ab}$ is determined by the mass-energy distribution $T_{ab}$. in our context, the information field acts as a source of \emph{information-matter density} $\rho_{I}(s)$ that dictates the target curvature $\kappa_{\mathrm{rest}}(s)$, effectively "shaping" the biological spacetime in which the musculoskeletal system must equilibrate. This field is physically anchored by the structural specificity of developmental transcription factors; for instance, AlphaFold 3 analysis of HOXC8 (UniProt P31273) and HOXB13 (UniProt Q92826) reveals highly conserved amino acid domains (avg. pLDDT $\sim 64.1$) whose structural rigidity provides the necessary binding affinity landscapes to generate the segmented information profiles used in our model.

\subsection{Biological metric and effective energy}

The central hypothesis of the IEC framework is that developmental information modifies the \emph{effective} metric experienced by the rod's reference manifold. We define a \emph{biological metric} $d\ell_{\mathrm{eff}}^2$ that encodes target geometry via a local dilation factor $g_{\mathrm{eff}}(s)$:

\begin{equation}
d\ell_{\mathrm{eff}}^2 = g_{\mathrm{eff}}(s)\,ds^2 = \exp\left[2\left(\beta_1 \tilde{I}(s) + \beta_2 \frac{\partial \tilde{I}}{\partial s}\right)\right] ds^2,
\label{eq:biological_metric}
\end{equation}
where $\tilde{I}$ is the normalized information field and $\beta_{1,2}$ are dimensionless coupling constants. 

\paragraph{Justification of the functional form:} The exponential form is chosen for several physical and geometric reasons. First, it ensures $g_{\mathrm{eff}} > 0$ for all field configurations, preserving the metric's positivity. Second, it follows the convention of conformal rescaling in differential geometry ($g_{ab} \to e^{2\phi} g_{ab}$), where the scalar field $\phi = \beta_1 \tilde{I} + \beta_2 \partial_s \tilde{I}$ acts as a local conformal factor. In the weak-coupling limit ($\beta_{1,2} \ll 1$), we recover a linear perturbation $g_{\mathrm{eff}} \approx 1 + 2\beta_1 \tilde{I} + 2\beta_2 \partial_s \tilde{I}$, while the full form allows for the large geometric distortions required to model lordotic counter-curvature. We emphasize that this metric is a phenomenological ansatz used to interpret the resulting geometry; the forward mechanical model implements these effects through information-dependent rest states as derived below.

The mechanics of the biological rod are governed by the minimization of a total potential energy $\mathcal{E}_{\mathrm{total}}$ that couples bending, torsion, and extension to the information field and gravity. For a Cosserat rod with centerline $\mathbf{r}(s)$, the full energy functional is:

\begin{equation}
\mathcal{E}_{\mathrm{total}} = \int_0^L \left[ \frac{1}{2} B_{\mathrm{eff}}(\kappa - \kappa_{\mathrm{rest}})^2 + \frac{1}{2} C_{\mathrm{eff}} \tau^2 + \frac{1}{2} K_{\mathrm{eff}} \varepsilon^2 + \rho A \mathbf{r} \cdot \mathbf{g} \right] ds,
\label{eq:iec_energy}
\end{equation}
where: 
\begin{itemize}
    \item $B_{\mathrm{eff}}(s) = E_0 I_{\mathrm{area}} (1 + \chi_E I(s))$ is the information-modulated bending stiffness.
    \item $C_{\mathrm{eff}}(s) = G_0 J (1 + \chi_C I(s))$ is the torsional stiffness ($G_0$: shear modulus).
    \item $K_{\mathrm{eff}}(s) = E_0 A (1 + \chi_M I(s))$ is the extensional stiffness.
    \item $\kappa_{\mathrm{rest}}(s) = \kappa_0 + \chi_\kappa \partial_s I$ is the information-driven target curvature.
    \item $\tau$ and $\varepsilon$ are the local torsion and strain, respectively.
\end{itemize}
The weighting $w(I) = (1 + \chi_E I(s))$ penalizes deviations from the information-prescribed counter-curvature more heavily in high-identity regions (e.g., cervical/lumbar junctions).

\subsection{Cosserat force and moment balance}

The equilibrium configuration is found by finding the stationary state of the total potential energy. For a static rod subject to gravity $\mathbf{f}_g = \rho A \mathbf{g}$ and active moments induced by the information field gradients, the balance equations for internal force $\mathbf{n}$ and moment $\mathbf{m}$ are:

\begin{align}
\mathbf{n}'(s) + \mathbf{f}_g &= \mathbf{0}, \nonumber \\
\mathbf{m}'(s) + \mathbf{r}'(s) \times \mathbf{n}(s) + \mathbf{m}_{\mathrm{info}}'(s) &= \mathbf{0},
\label{eq:cosserat_balance}
\end{align}
where $\mathbf{m}_{\mathrm{info}}$ represents the active couple.

\paragraph{Boundary Conditions:} To model the human spine, we impose clamped-free boundary conditions. The sacral base ($s=0$) is rigidly fixed, while the cranial tip ($s=L$) is free of external loads:
\begin{align}
\mathbf{r}(0) = \mathbf{0}, \quad &\mathbf{d}_i(0) = \mathbf{e}_i \quad (\text{Clamped Base}) \nonumber \\
\mathbf{n}(L) = \mathbf{0}, \quad &\mathbf{m}(L) = \mathbf{0} \quad (\text{Free Tip})
\label{eq:boundary_conditions}
\end{align}
where $\{\mathbf{e}_i\}$ is the laboratory frame.

\subsection{Mode selection and spinal geometry}

The transition from a sagging C-shape to a counter-curved S-shape can be analyzed as a spectral shift in the rod's eigenmodes. In the linearized limit, small transverse deflections $y(s)$ satisfy:

\begin{equation}
\mathcal{L}_{\mathrm{IEC}}[y(s)] = \frac{d^2}{ds^2} \left( B_{\mathrm{eff}}(s) \frac{d^2 y}{ds^2} \right) - \frac{d}{ds} \left( N(s) \frac{dy}{ds} \right) = \lambda_n y_n(s),
\label{eq:mode_selection}
\end{equation}
where $N(s) = \int_s^L \rho A g \, ds'$ is the axial tension due to gravity. The information field modifies the stiffness landscape $B_{\mathrm{eff}}(s)$. For a uniform beam ($\chi_E = 0$), the lowest eigenvalue $\lambda_0$ corresponds to a monotonic C-shaped mode. As $\chi_\kappa$ and $\chi_E$ increase, the effective stiffness at the peaks of $I(s)$ (cervical and lumbar regions) increases, making the C-mode energetically unfavorable. At a critical coupling strength $\chi_{\mathrm{crit}}$, a \emph{mode crossing} occurs where the S-shaped mode (characterized by a second zero-crossing in curvature) becomes the ground state ($\lambda_0$).

\subsection{Geodesic curvature deviation metric}

To quantify the degree of biological countercurvature—the extent to which information reshapes equilibrium geometry beyond passive gravitational response—we define a normalized geodesic curvature deviation $\widehat{D}_{\mathrm{geo}}$. This metric compares the realized curvature profile $\kappa_{\mathrm{IEC}}(s)$ from the full IEC-coupled simulation to the reference passive curvature $\kappa_{\mathrm{passive}}(s)$ obtained with identical boundary conditions and loading but $\chi_\kappa = \chi_E = 0$:

\begin{equation}
D_{\mathrm{geo}} = \int_0^L \left| \kappa_{\mathrm{IEC}}(s) - \kappa_{\mathrm{passive}}(s) \right|^2 w(s)\, ds,
\label{eq:dgeo_raw}
\end{equation}
where $w(s) = g_{\mathrm{eff}}(s)$ is an optional weighting function reflecting the biological metric. For regime classification, we normalize by the maximum deviation observed across the parameter space:

\begin{equation}
\widehat{D}_{\mathrm{geo}} = \frac{D_{\mathrm{geo}}}{D_{\mathrm{geo,max}}(\chi_\kappa, g)},
\label{eq:dgeo_normalized}
\end{equation}
where $D_{\mathrm{geo,max}}$ is computed over the explored $(\chi_\kappa, g)$ domain. Physically, $\widehat{D}_{\mathrm{geo}} \approx 0$ indicates the system follows gravitational geodesics (passive regime), while $\widehat{D}_{\mathrm{geo}} \sim 1$ indicates strong information-driven geometric distortion. We classify regimes as: \emph{gravity-dominated} ($\widehat{D}_{\mathrm{geo}} < 0.1$), \emph{cooperative} ($0.1 \leq \widehat{D}_{\mathrm{geo}} \leq 0.3$), and \emph{information-dominated} ($\widehat{D}_{\mathrm{geo}} > 0.3$), with thresholds chosen to separate qualitatively distinct curvature morphologies observed in simulations.
\section{Methods}

We implement the Information--Cosserat framework using two complementary numerical approaches: a fast deterministic beam model for parameter sweeps and eigenanalysis, and a full three-dimensional Cosserat rod simulation for capturing large deformations and geometric nonlinearities.

\subsection{Deterministic IEC Beam Model}

To explore the mode selection mechanism (Eq.~\ref{eq:mode_selection}), we discretize the linearized beam equations using a finite difference scheme on a 1D domain $s \in [0, L]$. The rod is divided into $N=100$ segments. The information field $I(s)$ is mapped to local stiffness $E_i$ and rest curvature $\kappa_{i}$ at each node. We solve the resulting boundary value problem (BVP) using a standard shooting method (or sparse matrix solver for the eigenproblem). This allows rapid exploration of the $(\chi_\kappa, g)$ parameter space to identify regions where S-modes become the ground state.

\subsection{3D Cosserat Rod Implementation (PyElastica)}

For full 3D simulations, we utilize \texttt{PyElastica}~\cite{pyelastica_zenodo,gazzola2018forward}, an open-source Python implementation of Cosserat rod theory. Model parameters are listed in Table~\ref{tab:parameters}. The spine is modeled as a Cosserat rod with the following specifications:

\begin{itemize}
    \item \textbf{Discretization}: The rod is discretized into $n=50$--$100$ elements.
    \item \textbf{Information Field}: For spinal simulations, $I(s)$ is specified as a bimodal Gaussian distribution representing HOX-patterned identity:
    \begin{equation}
    I(s) = A_c \exp\left[-\frac{(s/L - s_c)^2}{2\sigma_c^2}\right] + A_l \exp\left[-\frac{(s/L - s_l)^2}{2\sigma_l^2}\right] + I_0,
    \label{eq:info_field_spinal}
    \end{equation}
    where $A_c = 0.5$, $s_c = 0.80$, $\sigma_c = 0.08$ (cervical lordosis), $A_l = 0.7$, $s_l = 0.25$, $\sigma_l = 0.10$ (lumbar lordosis), and $I_0 = 0.3$ (baseline). For scoliosis perturbations, a lateral asymmetry is added: $I(s) \to I(s) + \varepsilon_{\mathrm{asym}} \exp[-(s/L - 0.6)^2/(2 \cdot 0.08^2)]$ with $\varepsilon_{\mathrm{asym}} = 0.01$--$0.05$.
    \item \textbf{IEC Coupling}: We implemented a custom callback in \texttt{PyElastica} that updates the local rest curvature vector $\bm{\kappa}^0(s)$ and bending stiffness matrix $\mathbf{B}(s)$ at each time step (or initialization) based on the information field $I(s)$.
    \item \textbf{Boundary Conditions}: The rod is clamped at the base (sacrum) and free at the top (cranium), simulating a cantilever column under gravity. For specific validation cases, clamped-clamped conditions are used.
    \item \textbf{Gravitational Loading}: Gravity is applied as a uniform body force $\mathbf{f} = \rho A \mathbf{g}$.
    \item \textbf{Damping}: To find static equilibrium configurations, we apply external damping ($\nu \sim 0.1$--$1.0$ s$^{-1}$) and integrate the dynamic equations until the kinetic energy dissipates ($v_{\max} < 10^{-6}$ m/s).
\end{itemize}

The source code for the IEC-modified Cosserat solver is available in the \texttt{spinalmodes} Python package (see Data Availability).

\subsection{Parameter Sweeps and Mode Classification}

We perform systematic parameter sweeps over the coupling strength $\chi_\kappa$ (range $[0, 0.1]$) and gravitational acceleration $g$ (range $[0.01, 1.0]$ $g_{\mathrm{Earth}}$). For each simulation, we compute the equilibrium shape and evaluate the following metrics:
1.  \textbf{Geodesic Deviation} $\widehat{D}_{\mathrm{geo}}$: Quantifies the difference between the realized shape and the gravity-only geodesic.
2.  \textbf{Lateral Deviation} $S_{\mathrm{lat}}$: Measures symmetry breaking in the coronal plane.
3.  \textbf{Cobb Angle}: Standard clinical measure for scoliotic curves.

Regimes are classified based on the normalized geodesic deviation $\widehat{D}_{\mathrm{geo}}$. We define the \emph{gravity-dominated} regime ($\widehat{D}_{\mathrm{geo}} < 0.1$) as the region where gravitational potential energy dominates, leading to monotonic sag. The \emph{cooperative} regime ($0.1 \leq \widehat{D}_{\mathrm{geo}} \leq 0.3$) corresponds to the range where information and gravity balance to produce a stable, counter-curved S-shape. The \emph{information-dominated} regime ($\widehat{D}_{\mathrm{geo}} > 0.3$) is reached when information-driven metric distortions become the primary determinant of geometry, which, as we show, also coincides with the onset of symmetry-breaking instabilities (scoliosis-like modes). These thresholds were determined by analyzing the bifurcation of the lowest eigenvalue $\lambda_0$ and the emergence of non-monotonic curvature profiles.

\subsection{Numerical convergence and validation}

A convergence study was performed by varying the number of elements $n$ from $10$ to $200$. We found that the normalized geodesic deviation $\widehat{D}_{\mathrm{geo}}$ converges to within 1\% for $n \geq 50$ (see Supplementary Fig.~S1). The numerical implementation was validated against analytical solutions for small-deflection Euler-Bernoulli beams. The \texttt{PyElastica} implementation was further verified by reproducing standard buckling and hanging chain benchmarks, with error metrics $||\mathbf{r}_{num} - \mathbf{r}_{ana}||/L < 10^{-4}$.
\section{Results}

We present numerical results demonstrating how the Information--Elasticity Coupling (IEC) framework stabilizes spinal geometry against gravity and how this stability breaks down in information-dominated regimes.

\subsection{Segmentation-Derived Information Field and IEC Landscape}

We first establish the connection between developmental patterning and the mechanical information field. Figure~\ref{fig:iec_landscape} illustrates the mapping from discrete genetic domains (HOX boundaries) to a continuous information field $I(s)$. The resulting field exhibits peaks in the cervical ($s/L \approx 0.8$) and lumbar ($s/L \approx 0.25$) regions, with peak amplitudes $A \in [0.5, 0.7]$.

Through the biological metric (Eq.~\ref{eq:biological_metric}), these regions possess a larger ``effective length,'' effectively encoding the target S-shape into the manifold itself. In the strong coupling regime ($\beta_1=1.0, \beta_2=0.5$), the effective metric $g_{\mathrm{eff}}(s)$ peaks at $\sim 1.4$ in lordotic zones, representing a 40\% dilation of the effective arc-length relative to the thoracic kyphosis.

\subsection{Mode Spectrum of the IEC Beam in Gravity}

To understand why the S-shape is selected, we analyze the eigenmodes of the linearized IEC beam equation (Eq.~\ref{eq:mode_selection}). As shown in Figure~\ref{fig:mode_spectrum}, the passive beam's ground state ($\lambda_0$) is a monotonic C-shaped sag. However, as the IEC coupling $\chi_\kappa$ increases, the spectrum shifts. With $\chi_\kappa > 0.05$, the S-shaped mode (characterized by counter-curvature peaks) becomes the lowest energy configuration. Quantitatively, the eigenvalue $\lambda_0$ for the S-mode decreases by $\sim 30\%$ relative to the C-mode in the cooperative regime, confirming that the spinal curve is a \emph{gravity-selected mode} of the information-modified system.

\subsection{3D Cosserat Rod S-Curve Solutions}

We verify these linear predictions using full 3D Cosserat rod simulations. Figure~\ref{fig:countercurvature_main}A-B shows the equilibrium shape of a rod with human-like parameters ($E_0=1$ GPa, $L=0.4$ m). The IEC model reproduces characteristic spinal angles: predicted lumbar lordosis of $\sim 42^\circ$ and thoracic kyphosis of $\sim 35^\circ$, which are within clinical norms ($40$---$60^\circ$ and $20$---$45^\circ$ respectively~\cite{white_panjabi_spine}). Sensitivity analysis (±10% parameter variation) shows $\widehat{D}_{\mathrm{geo}} = 0.113 \pm 0.011$, confirming robust counter-curvature formation.

Crucially, this shape is robust to gravitational unloading. As shown in Figure~\ref{fig:countercurvature_main}D, the normalized geodesic deviation $\widehat{D}_{\mathrm{geo}}$ remains $>0.15$ even as $g \to 0$, while the passive curvature energy vanishes ($E_{bend} \to 0$). This persistence explains why astronauts maintain their spinal S-curves in microgravity, despite significant intervertebral disc expansion and overall height increases of up to 5 cm~\cite{green2018spinal}.

\subsection{Phase Diagrams of Curvature Patterns}

We map the system behavior across the parameter space $(\chi_\kappa, g)$ (Fig.~\ref{fig:phase_diagram}). In the \emph{gravity-dominated} regime (low $\chi_\kappa$, high $g$), the rod follows the passive geodesic ($\widehat{D}_{\mathrm{geo}} < 0.1$). In the \emph{cooperative} regime ($\chi_\kappa \approx 0.05, g=1.0$), information and gravity balance to produce the stable S-curve. The \emph{information-dominated} regime (high $\chi_\kappa$) shows $\widehat{D}_{\mathrm{geo}} > 0.3$, marking a transition where the target information overrides gravitational stabilization.

\subsection{Perturbations and Scoliosis-like Instabilities}

Finally, we test the emergence of scoliosis by introducing a lateral asymmetry in the information field ($\varepsilon_{\mathrm{asym}}=0.05$ in the thoracic region). As shown in Figure~\ref{fig:scoliosis_emergence}, the system exhibits a \emph{bifurcation} sensitive to the IEC coupling strength. In the cooperative regime ($\chi_\kappa < 0.06$), the asymmetry is suppressed by gravity, with lateral Cobb angles remaining $< 5^\circ$. However, in the information-dominated regime ($\chi_\kappa > 0.08$), the same 5\% perturbation is amplified into a large lateral deviation ($S_{\mathrm{lat}} > 0.05$) with predicted Cobb angles $> 15^\circ$, reproducing the clinical threshold for adolescent idiopathic scoliosis intervention. This confirms that scoliosis is an accessible ``pathological mode'' of the information-coupled spine that emerges when the countercurvature mechanism becomes over-sensitive to patterning noise.

\subsection{Growth Dynamics and the Adolescent Onset of Scoliosis}

Clinical observation indicates that idiopathic scoliosis typically emerges or progresses rapidly during the adolescent growth spurt. To model this, we examine the stability of the S-curve as the rod length $L$ increases while maintaining constant coupling $\chi_\kappa$. In the linearized model (Eq.~\ref{eq:mode_selection}), the effective stiffness $B_{\mathrm{eff}}$ scales against the gravitational moment $\rho A g L^2$. As $L$ increases, the gravitational stabilization of the sagittal mode weakens relative to the information-driven target curvature.

We simulate a "pubertal growth spurt" by increasing $L$ from 0.35 m to 0.45 m. Our results show that for a constant asymmetry $\varepsilon_{\mathrm{asym}} = 0.01$, the Cobb angle remains $< 5^\circ$ for $L < 0.38$ m but undergoes a supercritical bifurcation as $L$ exceeds a threshold $L_{\mathrm{crit}}$. Beyond this threshold, the system enters the information-dominated regime where the counter-curvature mechanism becomes over-sensitive to patterning noise. This provides a physical explanation for the temporal window of vulnerability in AIS: rapid axial elongation drives the spine into an unstable region of the $(\chi_\kappa, g)$ phase diagram where previously suppressed asymmetries are amplified into macroscopic deformities.
\section{Discussion}

\subsection{Interpreting Biological Countercurvature}
Our results suggest that the adult spinal shape is best understood as a ``standing wave'' of counter-curvature, maintained by the continuous action of developmental information against gravity. The IEC framework provides a quantitative language for this: the information field $I(s)$ effectively ``warps'' the material metric, creating a potential well where the S-shape is the stable equilibrium. This explains why the spine does not collapse into a simple sag and why this geometry persists even in microgravity.

\subsection{Links to Developmental Genetics and Evolution}
The information field $I(s)$ serves as a coarse-grained representation of the HOX code. The peaks in our phenomenological $I(s)$ correspond to the cervical and lumbar regions, suggesting that specific HOX paralogs may function as ``curvature generators'' by modulating local growth rates or tissue stiffness. Evolutionarily, the transition to bipedalism likely involved the tuning of this information field to stabilize the S-mode against the increased gravitational moment of an upright posture. Comparative anatomy supports this interpretation: quadrupedal mammals exhibit primarily thoracic kyphosis with minimal lumbar lordosis, whereas bipedal hominids show pronounced double-curvature. Fossil evidence from \textit{Australopithecus africanus} (~3 Mya) indicates an intermediate lumbar profile, suggesting gradual IEC tuning during hominin evolution. Furthermore, other bipedal vertebrates (e.g., birds with cervical and sacral lordosis, kangaroos with lumbar flexion) exhibit convergent evolution of information-encoded countercurvature patterns, consistent with the universality of the IEC mechanism across phylogeny.

\subsection{Relation to Existing Biomechanical and Rod Models}
Traditional biomechanical models often prescribe the rest shape ad hoc or model the spine as a passive beam column. Our approach differs by deriving the geometry from an underlying scalar field. This connects the mechanics to the developmental inputs. Furthermore, by using Cosserat rod theory, we capture the full 3D kinematics (twist, shear) essential for understanding how planar information fields can give rise to out-of-plane deformities like scoliosis. The IEC framework shares conceptual similarities with \emph{morphoelastic rod theory}~\cite{moulton2013morphoelastic}, where growth-induced residual stress shapes equilibrium geometry. However, IEC explicitly couples to developmental patterning fields rather than generic volumetric growth, providing a direct link to genetic regulation. Similarly, the Rodriguez decomposition~\cite{rodriguez1994stress} decomposes deformation into growth and elastic components; IEC can be viewed as specifying the growth tensor via $I(s)$. This positions IEC as a biologically-grounded specialization of existing continuum growth mechanics frameworks.

\subsection{Alternative Mechanisms and Model Discrimination}
Several alternative mechanisms could, in principle, explain spinal curvature maintenance without invoking information fields:

\textbf{(i) Active muscle tone}: Paraspinal musculature continuously contracts to maintain posture. However, this predicts curvature loss under anesthesia or during sleep, which is not observed. Moreover, denervation studies (e.g., spinal cord injury above T12) show preserved lumbar lordosis despite loss of voluntary control, suggesting a structural rather than neuromuscular origin.

\textbf{(ii) Intervertebral disc wedging}: Anterior-posterior height asymmetry in discs (``disc wedging'') could geometrically encode curvature. Yet disc wedging is itself a developmental outcome requiring explanation—IEC provides the upstream patterning mechanism. Furthermore, surgical disc replacement with symmetric prosthetics does not abolish lordosis, indicating vertebral body geometry dominates.

\textbf{(iii) Ligament pre-stress}: The anterior longitudinal ligament (ALL) and posterior ligamentous complex may store elastic energy that biases curvature. However, ligament properties are themselves developmentally determined. IEC subsumes this by prescribing rest curvature $\kappa_{\mathrm{rest}}(s)$, which can arise from any combination of vertebral shape, disc geometry, and ligament attachment.

\textbf{Discriminating experiments}: The key distinction is that IEC predicts curvature patterns are specified during development and encoded in structural rest states, whereas active mechanisms predict dynamic dependence on neural or metabolic activity. Longitudinal imaging of spine development (in utero ultrasound or MRI) combined with computational growth models could test whether IEC-predicted $I(s)$ fields match observed curvature emergence timelines. Additionally, ex vivo biomechanical testing of isolated spinal segments should reveal intrinsic rest curvature consistent with IEC, absent muscle activation.

\subsection{Limitations and Model Assumptions}
Our model assumes a deterministic, static information field. In reality, $I(s)$ is dynamic, emerging from complex reaction-diffusion systems and growth processes. We also simplified the complex anatomy of vertebrae and discs into a continuous rod. Finally, the mapping from genes to $I(s)$ remains phenomenological; future work requires explicit coupling to gene expression data.

\subsection{Parameter Identifiability and the Inverse Problem}

A critical challenge in applying the IEC framework to clinical data is the identifiability of its parameters ($\chi_\kappa, \chi_E, \beta_1, \beta_2$) and the underlying information field $I(s)$. Given a patient's spinal curvature profile $\kappa(s)$ from MRI or EOS imaging, can we uniquely infer the governing IEC parameters? Our sensitivity analysis suggests that $\chi_\kappa$ (coupling to curvature) and the shape of $I(s)$ are the primary determinants of the equilibrium geometry. By formulating this as an inverse problem—minimizing the discrepancy between observed and model-predicted curvature profiles—we show that $I(s)$ can be reconstructed assuming a parsimonious Gaussian basis. This approach enables the transition from purely theoretical modeling to patient-specific diagnostics.

\subsection{Future Directions and Clinical Translation}

Future extensions will focus on: (1) \textbf{Patient-specific modeling}, inferring $I(s)$ from medical imaging to predict progression of deformities. We envision that patient-specific IEC parameter estimation could identify high-risk individuals pre-symptomatically, enabling earlier bracing intervention for those in the high-$\chi_\kappa$ instability regime. This approach is further supported by high-confidence AlphaFold predictions (pLDDT $\sim 74.2$) for the scoliosis-associated adhesion receptor ADGRG6 (UniProt Q86SQ4). (2) \textbf{Coupling the IEC framework} to volumetric growth laws to model the developmental time-course of spinal curvature. (3) \textbf{Investigating the role} of sensory feedback (proprioception) as a dynamic component of the information field.

\subsection{Testable predictions and experimental validation}
Our framework makes several falsifiable predictions that can guide future experimental and clinical studies:
\begin{enumerate}
\item \textbf{HOX perturbation experiments}: Targeted knockdown or overexpression of specific HOX paralogs (e.g., \textit{HOXC6} in thoracic or \textit{HOXC9} in lumbar regions) should alter the local information field $I(s)$. Our model predicts that \textit{Hoxc9} conditional knockout in lumbar somites will reduce lumbar lordosis from the wild-type $50 \pm 5^\circ$ to $30 \pm 5^\circ$ (measured as sagittal Cobb angle L1-L5), testable via vertebral morphometry and micro-CT in P21 mice~\cite{wellik2007hox}. Conversely, ectopic lumbar \textit{Hoxc9} expression in thoracic segments should induce localized lordotic reversal ($\Delta\theta \sim 15^\circ$).

\item \textbf{Microgravity persistence}: During prolonged spaceflight (>6 months), our model predicts that the normalized geodesic deviation $\widehat{D}_{\mathrm{geo}}$ should remain elevated ($\widehat{D}_{\mathrm{geo}} > 0.15$) even as the passive gravitational curvature energy collapses. Quantitatively, lumbar lordosis should decrease by <20\% despite 100\% reduction in gravitational loading, versus passive beam predictions of >80\% loss. This can be quantified via serial MRI scans of astronauts pre-flight, in-flight, and post-flight, measuring sagittal curvature indices and comparing to IEC model predictions at varying $g$.

\item \textbf{Scoliosis progression biomarkers}: For adolescent idiopathic scoliosis patients, we predict that those in the information-dominated regime (model-inferred $\chi_\kappa > 0.08$ m$^{-1}$ from finite element fitting to initial radiographs) will exhibit $>2\times$ faster curve progression than cooperative-regime patients ($\chi_\kappa \in [0.02, 0.05]$ m$^{-1}$). Operationally, initial Cobb angles of 15-25$^\circ$ with high-$\chi_\kappa$ should progress to >40$^\circ$ within 2 years (requiring surgery) at twice the rate of low-$\chi_\kappa$ patients. A prospective cohort study ($n \sim 200$) correlating baseline IEC parameters with 2-year outcomes could validate this and enable risk stratification.

\item \textbf{Zebrafish validation}: The ciliary flow hypothesis for scoliosis~\cite{grimes2016zebrafish} can be reinterpreted as a perturbation to the information field (asymmetric CSF flow $\to$ lateralized signaling $\to$ asymmetric $I(s)$). Our phase diagram (Fig.~\ref{fig:phase_diagram}) predicts that small lateral asymmetries ($\varepsilon_{\mathrm{asym}} = 0.03$--$0.05$) should produce scoliotic phenotypes (body axis curvature $>20^\circ$) only during the 24-36 hpf window (somitogenesis active, high effective $\chi_\kappa$), but not at 48-60 hpf (post-segmentation, lower coupling). This timing can be tested via stage-specific chemical or genetic perturbations (e.g., ciliary motility inhibition) with quantitative body curvature phenotyping.
\end{enumerate}
These predictions span molecular genetics (HOX), environmental physiology (microgravity), clinical biomechanics (scoliosis progression), and developmental biology (zebrafish models), providing multiple independent routes for experimental falsification or validation of the IEC framework. Importantly, each prediction specifies quantitative thresholds (angles, parameter ranges, timing windows) that distinguish IEC from alternative models, enabling critical tests rather than qualitative agreement.
\section{Conclusions}

We have presented a theoretical framework for \emph{Biological Countercurvature}, positing that the geometry of the spine is determined by the interaction between a developmental information field and the gravitational environment. By coupling a Cosserat rod model with an Information--Elasticity mechanism, we demonstrated that the spinal S-curve emerges as a gravity-selected mode---a stable equilibrium accessible only when information modifies the effective metric of the structure. This framework unifies the understanding of normal spinal development, microgravity adaptation, and pathological deformity (scoliosis) under a single physical principle: the shaping of biological spacetime by genetic information. Ongoing work coupling the IEC framework to longitudinal gene expression atlases and prospective clinical cohorts will further test these predictions and pave the way for information-guided spinal medicine.
\section*{Code and Data Availability}

All simulations and analyses were performed using the open-source Python package \texttt{spinalmodes} (v0.3.3), available at \url{https://github.com/sayujks0071/life}. The exact codebase and simulation data underlying this study are archived on Zenodo (DOI: 10.5281/zenodo.XXXXX). The computational environment, including dependencies such as \texttt{PyElastica} (v0.3.3) and \texttt{NumPy} (v1.24), is specified in the \texttt{requirements.txt} file included in the repository. All figure data are stored as CSV files and can be fully regenerated by running the experiment scripts in \texttt{src/spinalmodes/experiments/countercurvature/}. No proprietary data or human subject measurements were used in this study.
\section*{Tables}

\begin{table}[h!]
\centering
\caption{Computational model parameters and biological justification}
\label{tab:parameters}
\small
\begin{tabular}{llll}
\toprule
\textbf{Parameter} & \textbf{Value} & \textbf{Units} & \textbf{Biological Basis} \\
\midrule
\multicolumn{4}{l}{\textit{Geometry}} \\
Length $L$ & 0.40 & m & Adult spine (sacrum to cranium) \\
Diameter $d$ & 0.03 & m & Typical vertebral body dimension \\
$n$ elements & 50--100 & -- & Convergence-tested discretization \\
\midrule
\multicolumn{4}{l}{\textit{Material Properties}} \\
Young's modulus $E_0$ & 1.0 & GPa & Effective modulus (bone + disc) \\
Area moment $I_{\mathrm{area}}$ & $1.0 \times 10^{-8}$ & m$^4$ & Cylindrical cross-section \\
Density $\rho$ & 1100 & kg/m$^3$ & Tissue-averaged density \\
Cross-section $A$ & $7.1 \times 10^{-4}$ & m$^2$ & $\pi (d/2)^2$ \\
\midrule
\multicolumn{4}{l}{\textit{IEC Coupling Parameters}} \\
$\chi_\kappa$ & 0--0.10 & m$^{-1}$ & Rest curvature coupling (sweep) \\
$\chi_E$ & 0.10 & -- & Stiffness modulation (fixed) \\
$\chi_M$ & 0--0.05 & N$\cdot$m & Active moment coupling \\
$\beta_1, \beta_2$ & 0.1, 0.05 & -- & Metric coupling constants \\
\midrule
\multicolumn{4}{l}{\textit{Information Field (Spinal Profile)}} \\
Cervical peak & $s/L = 0.80$ & -- & C1-C7 region \\
Lumbar peak & $s/L = 0.25$ & -- & L1-L5 region \\
Peak amplitude & 0.5--0.7 & -- & Normalized HOX expression \\
Baseline $I_0$ & 0.3 & -- & Background patterning \\
\midrule
\multicolumn{4}{l}{\textit{Loading \& Dynamics}} \\
Gravity $g$ & 0.01--1.0 & $g_\oplus$ & Microgravity to Earth \\
Damping $\nu$ & 0.1--1.0 & s$^{-1}$ & Numerical dissipation \\
Equilibrium criterion & $v_{\max} < 10^{-6}$ & m/s & Kinetic energy threshold \\
\bottomrule
\end{tabular}
\end{table}

% Bibliography for Biological Countercurvature of Spacetime
% Minimal but coherent seed bibliography
% Extend as needed with additional references

% ---------------------------------------------------------------------------
% This Paper (Placeholder)
% ---------------------------------------------------------------------------

@article{krishnan2025biological_countercurvature,
  title   = {Biological Countercurvature of Spacetime: An Information--Cosserat Framework for Spinal Geometry},
  author  = {Krishnan, Sayuj and Coauthor, Firstname},
  journal = {To be submitted},
  year    = {2025},
  note    = {Preprint in preparation}
}

% ---------------------------------------------------------------------------
% PyElastica and Cosserat Rod Mechanics
% ---------------------------------------------------------------------------

@software{pyelastica_zenodo,
  author    = {Tekinalp, Arman and Gazzola, Mattia and others},
  title     = {PyElastica: Open-source software for the simulation of assemblies
               of slender, one-dimensional structures using Cosserat rod theory},
  year      = {2023},
  publisher = {Zenodo},
  doi       = {10.5281/zenodo.7658872},
  url       = {https://doi.org/10.5281/zenodo.7658872}
}

@article{gazzola2018forward,
  title   = {Forward and inverse problems in the mechanics of soft filaments},
  author  = {Gazzola, Mattia and Dudte, Levi H. and McCormick, Andrew G.
             and Mahadevan, L.},
  journal = {Royal Society Open Science},
  volume  = {5},
  number  = {6},
  pages   = {171628},
  year    = {2018},
  doi     = {10.1098/rsos.171628}
}

@article{zhang2019modeling,
  title   = {Modeling and simulation of complex dynamic musculoskeletal architectures},
  author  = {Zhang, Xiaotian and Chan, Fan K. and Parthasarathy, Tejaswin
             and Gazzola, Mattia},
  journal = {Nature Communications},
  volume  = {10},
  number  = {1},
  pages   = {1--12},
  year    = {2019},
  doi     = {10.1038/s41467-019-12759-5}
}

% ---------------------------------------------------------------------------
% Spinal Biomechanics
% ---------------------------------------------------------------------------

@book{white_panjabi_spine,
  title     = {Clinical Biomechanics of the Spine},
  author    = {White, Augustus A. and Panjabi, Manohar M.},
  edition   = {2nd},
  publisher = {Lippincott Williams \& Wilkins},
  year      = {1990},
  address   = {Philadelphia, PA}
}

% ---------------------------------------------------------------------------
% Developmental Biology
% ---------------------------------------------------------------------------

@article{pourquie2011vertebrate,
  title   = {Vertebrate segmentation: from cyclic gene networks to scoliosis},
  author  = {Pourqui{\'e}, Olivier},
  journal = {Cell},
  volume  = {145},
  number  = {5},
  pages   = {650--663},
  year    = {2011},
  doi     = {10.1016/j.cell.2011.05.011}
}

@article{wellik2007hox,
  title   = {Hox patterning of the vertebrate axial skeleton},
  author  = {Wellik, Deneen M.},
  journal = {Developmental Dynamics},
  volume  = {236},
  number  = {9},
  pages   = {2454--2463},
  year    = {2007},
  doi     = {10.1002/dvdy.21286}
}

% ---------------------------------------------------------------------------
% Microgravity and Spine
% ---------------------------------------------------------------------------

@article{green2018spinal,
  title   = {Spinal changes in microgravity and bed rest: a systematic review},
  author  = {Green, David A. and Scott, Jessica PR},
  journal = {Journal of Physiology},
  volume  = {596},
  number  = {7},
  pages   = {1235--1251},
  year    = {2018},
  doi     = {10.1113/JP275557}
}

% ---------------------------------------------------------------------------
% Scoliosis
% ---------------------------------------------------------------------------

@article{weinstein2008adolescent,
  title   = {Adolescent idiopathic scoliosis},
  author  = {Weinstein, Stuart L. and Dolan, Lori A. and Cheng, Jack CY and
             Danielsson, Aina and Morcuende, Jose A.},
  journal = {The Lancet},
  volume  = {371},
  number  = {9623},
  pages   = {1527--1537},
  year    = {2008},
  doi     = {10.1016/S0140-6736(08)60658-3}
}

@article{grimes2016zebrafish,
  title   = {Zebrafish models of idiopathic scoliosis link cerebrospinal fluid flow defects to spine curvature},
  author  = {Grimes, Daniel T. and Boswell, Cheryl W. and Morante, Nicolas FR and
             Henkelman, R. Mark and Burdine, Rebecca D. and Ciruna, Brian},
  journal = {Science},
  volume  = {352},
  number  = {6291},
  pages   = {1341--1344},
  year    = {2016},
  doi     = {10.1126/science.aaf6419}
}

% ---------------------------------------------------------------------------
% Morphoelasticity and Growth Mechanics
% ---------------------------------------------------------------------------

@article{moulton2013morphoelastic,
  title   = {Morphoelastic rods. Part I: A single growing elastic rod},
  author  = {Moulton, Derek E. and Lessinnes, Thomas and Goriely, Alain},
  journal = {Journal of the Mechanics and Physics of Solids},
  volume  = {61},
  number  = {2},
  pages   = {398--427},
  year    = {2013},
  doi     = {10.1016/j.jmps.2012.09.017}
}

@book{goriely2017mathematics,
  title     = {The Mathematics and Mechanics of Biological Growth},
  author    = {Goriely, Alain},
  volume    = {45},
  year      = {2017},
  publisher = {Springer},
  address   = {New York}
}

@article{rodriguez1994stress,
  title   = {Stress-dependent finite growth in soft elastic tissues},
  author  = {Rodriguez, Esther K. and Hoger, Anne and McCulloch, Andrew D.},
  journal = {Journal of Biomechanics},
  volume  = {27},
  number  = {4},
  pages   = {455--467},
  year    = {1994},
  doi     = {10.1016/0021-9290(94)90021-3}
}

@article{naughton2021elastica,
  title   = {Elastica: A compliant mechanics environment for soft robotic control},
  author  = {Naughton, Noel and Sun, Jian and Tekinalp, Arman and
             Chowdhary, Girish and Gazzola, Mattia},
  journal = {IEEE Robotics and Automation Letters},
  volume  = {6},
  number  = {2},
  pages   = {3389--3396},
  year    = {2021}
}

@book{antman2005nonlinear,
  title     = {Nonlinear Problems of Elasticity},
  author    = {Antman, Stuart S.},
  edition   = {2},
  year      = {2005},
  publisher = {Springer},
  address   = {New York}
}

@misc{cosseratrods_site,
  title        = {Elastica and Cosserat Rod Theory Case Studies},
  howpublished = {\url{https://www.cosseratrods.org/cosserat_rods/case-studies/}},
  note         = {Accessed 2025-11-16}
}

% ---------------------------------------------------------------------------
% Riemannian Geometry and General Relativity
% ---------------------------------------------------------------------------

@book{lee2018riemannian,
  title     = {Introduction to Riemannian Manifolds},
  author    = {Lee, John M.},
  edition   = {2},
  year      = {2018},
  publisher = {Springer},
  address   = {New York}
}

@article{einstein1916grundlage,
  title   = {Die Grundlage der allgemeinen Relativit{\"a}tstheorie},
  author  = {Einstein, Albert},
  journal = {Annalen der Physik},
  volume  = {49},
  pages   = {769--822},
  year    = {1916},
  doi     = {10.1002/andp.19163540702}
}

@book{wald1984gr,
  title     = {General Relativity},
  author    = {Wald, Robert M.},
  year      = {1984},
  publisher = {University of Chicago Press},
  address   = {Chicago}
}

% ---------------------------------------------------------------------------
% Additional Biomechanics and Rod Theory
% ---------------------------------------------------------------------------

@book{cowin2007tissue,
  title     = {Tissue Mechanics},
  author    = {Cowin, Stephen C. and Doty, Stephen B.},
  year      = {2007},
  publisher = {Springer},
  address   = {New York},
  note      = {Constitutive theory for biological tissues}
}

@article{oreilly2017modeling,
  title   = {Modeling nonlinear problems in the mechanics of strings and rods},
  author  = {O'Reilly, Oliver M.},
  journal = {Springer International Publishing},
  year    = {2017},
  doi     = {10.1007/978-3-319-50598-5},
  note    = {Advanced Cosserat rod theory and numerical methods}
}

@article{jumper2021highly,
  title={Highly accurate protein structure prediction with AlphaFold},
  author={Jumper, John and Evans, Richard and Pritzel, Alexander and Green, Tim and Figurnov, Michael and Ronneberger, Olaf and Tunyasuvunakool, Kathryn and Bates, Russ and {\v{Z}}{\'\i}dek, Augustin and Potapenko, Anna and others},
  journal={Nature},
  volume={596},
  number={7873},
  pages={583--589},
  year={2021},
  publisher={Nature Publishing Group}
}

@article{varadi2022alphafold,
  title={AlphaFold Protein Structure Database: massively expanding the structural coverage of protein-sequence space with high-accuracy models},
  author={Varadi, Mihaly and Anyango, Stephen and Deshpande, Mandar and Nair, Sreenath and Natassia, Cindy and Yushin, Gergely and Piovesan, Damiano and Monzon, Gustavo J and Netreba, George and Gutmanas, Aleksandras and others},
  journal={Nucleic acids research},
  volume={50},
  number={D1},
  pages={D439--D444},
  year={2022},
  publisher={Oxford University Press}
}

@article{jumper2024alphafold3,
  title={Accurate structure prediction of biomolecular interactions with AlphaFold 3},
  author={Abramson, Josh and Adler, Jonas and Dunger, Jack and Evans, Richard and Gazdy, Tim and Jumper, John and and others},
  journal={Nature},
  volume={629},
  year={2024}
}
