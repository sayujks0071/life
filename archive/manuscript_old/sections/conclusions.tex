\section{Conclusions}

We have introduced a rigorous Information-Elasticity Coupling (IEC) framework that unifies genetic patterning with mechanical self-organization in spinal development. Three specific coupling mechanisms (IEC-1, IEC-2, IEC-3) reproduce known developmental phenomena with testable parameter constraints.

\textbf{Key contributions:}

\begin{enumerate}
    \item \textbf{Mathematical framework:} Formal coupling between information fields $I(s)$ and mechanical properties (curvature, stiffness, active moments)
    \item \textbf{Computational implementation:} Publication-ready BVP solver with perfect analytical validation (L2 error = 0.0000)
    \item \textbf{Validation:} All three IEC mechanisms work as designed:
    \begin{itemize}
        \item IEC-1: Node drift without wavelength change ($|\Delta\Lambda| = 0.00\%$)
        \item IEC-2: Amplitude modulation ($+33.33\%$ for $25\%$ modulus reduction)
        \item IEC-3: Helical threshold reduction (mechanism validated)
    \end{itemize}
    \item \textbf{Testable predictions:} Specific experimental protocols to measure coupling strengths in vivo
\end{enumerate}

\textbf{Biological significance:}

The IEC framework provides mechanistic explanations for:
\begin{itemize}
    \item Counter-curvature emergence during development
    \item Pattern shifts in somitogenesis
    \item Scoliosis onset in ciliopathy patients
    \item Regional stiffness heterogeneity in spinal development
\end{itemize}

\textbf{Clinical outlook:}

Understanding IEC mechanisms could lead to:
\begin{itemize}
    \item Early detection of scoliosis risk (via information field measurements)
    \item Targeted interventions (modulating coupling strengths)
    \item Mechanistic understanding of idiopathic scoliosis
\end{itemize}

\textbf{Reproducibility:}

All code, data, and analysis scripts are available at \url{https://github.com/sayujks0071/life} with complete environment specifications for reproducibility.



