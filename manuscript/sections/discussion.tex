\section{Discussion}

\subsection{Countercurvature regimes}

The phase diagram quantifies how biological information and gravity interact. In gravity-dominated regimes, the rod follows gravity-selected geodesics; information plays little role. In cooperative regimes, information reshapes curvature within the gravitational background. In information-dominated regimes, countercurvature governs the geometry and enables symmetry breaking. The normalized geodesic curvature deviation $\widehat{D}_{\mathrm{geo}}$ provides a quantitative measure of these interactions and transitions.

\subsection{Growth against gravity as a standing mode}

The adult sagittal spine can be interpreted as a standing counter-curvature mode selected by an information field acting against gravity, not as a passive beam sagging under load. As coupling increases, the system transitions from a C-shaped profile to a robust, sine-like S-curve that persists when gravity is reduced. This view extends to development (progressive recruitment of higher curvature modes) and to pathology, where the same machinery can amplify small asymmetries into lateral branches when information dominates.

\subsection{Analog gravity interpretation}

``Countercurvature of spacetime'' here is analog rather than fundamental: the Cosserat rod in a uniform gravitational field is the effective spacetime, and $I(s)$ modifies $d\ell_{\mathrm{eff}}^{2}=g_{\mathrm{eff}}(s)\,ds^{2}$. The quantity $\widehat{D}_{\mathrm{geo}}$ measures how strongly information reshapes equilibrium geometry relative to gravity-selected solutions, in analogy with additional fields modifying geodesics in general relativity~\cite{einstein1916grundlage,wald1984gr}. We do not propose any modification of Einstein's equations; the analog language organizes how developmental and neuromuscular information can select, stabilize, or destabilize curvature modes of the spine in a gravitational background.

\subsection{Implications for scoliosis and control}

Scoliosis-like patterns arise when countercurvature dominates: small asymmetries are amplified into lateral deviations. This quantifies how developmental or neuromuscular asymmetries might yield pathological curvature. The phase diagram suggests such behavior when information--curvature coupling is strong and gravity is moderate or reduced. Normal sagittal curvature and scoliosis-like deviations then appear as regimes of the same IEC mechanism, suggesting that interventions could target the coupling itself rather than treat them as separate phenomena.

Ciliary flow patterns provide a concrete biological example of information fields that can break left--right symmetry: coordinated ependymal cell cilia beating generates cerebrospinal fluid (CSF) flow gradients that establish spatial information fields~\cite{grimes2016zebrafish}. Disruptions in ciliary function lead to abnormal CSF flow patterns and are associated with elevated scoliosis incidence, consistent with the IEC framework's prediction that asymmetric information gradients can amplify into pathological curvature in the information-dominated regime.

\section{Limitations and Outlook}

The information field $I(s)$ is phenomenological; its form and couplings are chosen to match observed curvature. The countercurvature metric $g_{\mathrm{eff}}(s)$ is heuristic, with empirical weights $\beta_{1},\beta_{2}$. Most experiments use a simplified beam; full 3D Cosserat models are applied mainly to the scoliosis analysis and should be extended. The normalized geodesic curvature deviation $\widehat{D}_{\mathrm{geo}}$ can inflate as $g\to0$ because the passive energy denominator collapses; this is expected but merits care in the microgravity limit. In 2D beam models, $S_{\mathrm{lat}}$ and Cobb-like angles use a pseudo-coronal projection; full 3D rods provide the true coronal plane.

Future work includes: (1) applying the framework to experimental microgravity and clinical scoliosis data; (2) deriving $I(s)$ from developmental or neural control principles (e.g., HOX/PAX patterning~\cite{wellik2007hox} and cilia-driven flows~\cite{grimes2016zebrafish}); (3) relating information--curvature coupling to known biological processes in spinal development and control; and (4) exploring therapies that target the coupling itself.

\section{Conclusion}

We present a quantitative framework for biological countercurvature that unifies normal sagittal curvature and scoliosis-like deviations within a single IEC model. A biological metric $d\ell_{\mathrm{eff}}^{2}=g_{\mathrm{eff}}(s)\,ds^{2}$, a normalized geodesic curvature deviation $\widehat{D}_{\mathrm{geo}}$, and a phase diagram of countercurvature regimes together show that information-driven structure maintenance persists in microgravity and that normal and pathological patterns are regimes of the same model. The analog-gravity perspective---treating a rod in gravity as an effective spacetime and information fields as sources of countercurvature---links information processing, mechanics, and geometry in living systems. Extending this framework to data, first-principles information fields, and therapeutic strategies targeting information--curvature coupling are natural next steps.
