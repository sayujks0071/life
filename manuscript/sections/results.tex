\section{Results}

\subsection{Gravity-selected versus information-selected curvature modes}

In the spine-like configuration, the information field $I(s)$ produces a smooth S-shaped curvature profile $\kappa_{I}(s)$ that is well approximated by a single sign change along the cranio--caudal axis, whereas the purely gravity-selected solution $\kappa_{0}(s)$ tends toward a monotonic C-shaped sag. The stabilized sagittal S-curve is dominated by a single smooth sign-changing mode: $\kappa_{I}(s)$ exhibits only one sign change along the axis and a max-to-RMS curvature ratio of $\approx1.81$, consistent with a sine-like counter-curvature profile against gravity. The normalized geodesic curvature deviation between the gravity-selected and information-selected solutions is $\widehat{D}_{\mathrm{geo}}\approx0.14$, confirming that the information-driven S-curve is not a small perturbation of the passive sag. In this sense, the mature spine behaves as a sinusoidal counter-curvature mode stabilized by IEC.

For the plant-like configuration, varying $\chi_{\kappa}$ drives a transition from passive sag to active upward bending. The geodesic deviation $\widehat{D}_{\mathrm{geo}}$ quantifies this transition, with $\widehat{D}_{\mathrm{geo}}<0.1$ denoting gravity-dominated sag and $\widehat{D}_{\mathrm{geo}}>0.2$ indicating strong information-driven upward bending.

\subsection{Persistence of information-driven shape in microgravity}

Reducing gravity collapses passive curvature energy, yet the information-selected structure persists. As $g$ decreases from $1.0$ to $0.10$ (in Earth units), the normalized geodesic curvature deviation remains essentially unchanged, $\widehat{D}_{\mathrm{geo}}\approx0.091$ (changing by less than 1\% in our simulations), indicating that the information-selected ``spinal wave'' is geometrically stable in microgravity even as the passive response to gravity weakens~\cite{green2018spinal,marfia2023microgravity}. This persistence provides quantitative support for the biological countercurvature hypothesis: information fields can maintain structure even when gravitational loading is negligible.

\subsection{Phase diagram of countercurvature regimes}

We map countercurvature behavior across the $(\chi_{\kappa},g)$ parameter space, where $\chi_{\kappa}$ controls information-to-curvature coupling and $g$ denotes gravitational acceleration. The normalized geodesic deviation $\widehat{D}_{\mathrm{geo}}$ separates distinct regimes: gravity-dominated ($\widehat{D}_{\mathrm{geo}}<0.1$), cooperative ($0.1<\widehat{D}_{\mathrm{geo}}<0.3$), and information-dominated/scoliotic ($\widehat{D}_{\mathrm{geo}}>0.3$). In the present sweep we see gravity-dominated points with $\widehat{D}_{\mathrm{geo}}\approx0.059$ and negligible lateral indices (e.g., $\chi_{\kappa}=0.015$, $g=9.81$) and cooperative points with $\widehat{D}_{\mathrm{geo}}\approx0.15$ and visibly reshaped sagittal curvature (e.g., $\chi_{\kappa}=0.065$, $g=9.81$). Our thresholds for a scoliosis regime ($S_{\mathrm{lat}}\gtrsim0.05$, Cobb-like angles $\gtrsim5^{\circ}$) are not crossed in this parameter window; the symmetry-broken branch remains a predicted extension at larger $\chi_{\kappa}$ or stronger asymmetry rather than a realized regime in the current sweep.

\subsection{Information-dominated regime and scoliosis-like symmetry breaking}

To test left--right asymmetries, we add a small thoracic bump ($\varepsilon_{\mathrm{asym}}\approx5\%$) to $I(s)$ or the lateral rest curvature. Symmetric ($\varepsilon_{\mathrm{asym}}=0$) and asymmetric runs are simulated over $(\chi_{\kappa},g)$. From coronal projections we compute $S_{\mathrm{lat}}$ and Cobb-like angles, alongside $\widehat{D}_{\mathrm{geo}}$.

In gravity-dominated regions (low $\chi_{\kappa}$, high $g$), symmetric and asymmetric solutions are nearly identical: $S_{\mathrm{lat}}$ and Cobb-like angles change by only a few percent or degrees, and $\widehat{D}_{\mathrm{geo}}$ remains small. The perturbation is effectively suppressed by gravity-selected curvature. As we move toward the information-dominated corner of parameter space (high $\chi_{\kappa}$ at moderate or reduced $g$), the model predicts that the same perturbation can produce pronounced lateral deformation: $S_{\mathrm{lat}}\gtrsim0.05$, Cobb-like angles $\gtrsim5$--$10^{\circ}$, and $\widehat{D}_{\mathrm{geo}}$ in the large-deviation regime. In this regime, the information field reshapes the effective metric so strongly that a small asymmetry is expected to be amplified into a scoliosis-like branch. Thus scoliosis-like patterns can arise when countercurvature dominates gravity, without invoking a fundamentally different mechanical mechanism~\cite{weinstein2008adolescent,white_panjabi_spine}.
