\section{Theory: Information--Cosserat Model of Spinal Countercurvature}

We propose that the robust S-shaped geometry of the spine arises not from passive mechanical equilibrium under gravity, but from an active \emph{counter-curvature} mechanism driven by developmental information. We formalize this using an Information--Elasticity Coupling (IEC) framework, where a scalar information field $I(s)$ modifies the effective geometry and energetics of a Cosserat rod.

\subsection{Geometry and parameterization}

Consider a slender rod parameterized by arc-length $s \in [0, L]$. The configuration is defined by a centerline curve $\mathbf{r}(s) \in \mathbb{R}^3$ and a director frame $\{\mathbf{d}_1, \mathbf{d}_2, \mathbf{d}_3\}(s)$ describing the orientation of cross-sections. The rod deforms under a gravitational field $\mathbf{g} = -g \hat{\mathbf{e}}_z$. In the absence of biological regulation, such a rod would sag into a C-shape (kyphosis) or buckle.

\subsection{Information field from developmental patterning}

We introduce a scalar field $I(s)$ representing the spatial distribution of developmental identity along the axis (e.g., HOX gene expression domains or segmentation clock outputs). This field acts as a ``morphogenetic coordinate,'' encoding the target geometry. For the spine, $I(s)$ is modeled as a bimodal distribution peaking in the cervical and lumbar regions, corresponding to the lordotic curves required for upright posture.

\subsection{The Information--Cosserat Manifold}

The Human spine is modeled as a one-dimensional Cosserat rod embedded in three-dimensional Euclidean space. However, in the IEC framework, we treat the rod's reference configuration not as a flat segment, but as a manifold whose intrinsic geometry is warped by a developmental information field $I(s)$. We draw a formal analogy to General Relativity (GR), where the geometry of spacetime $g_{ab}$ is determined by the mass-energy distribution $T_{ab}$. in our context, the information field acts as a source of \emph{information-matter density} $\rho_{I}(s)$ that dictates the target curvature $\kappa_{\mathrm{rest}}(s)$, effectively "shaping" the biological spacetime in which the musculoskeletal system must equilibrate. This field is physically anchored by the structural specificity of developmental transcription factors; for instance, AlphaFold 3 analysis of HOXC8 (UniProt P31273) and HOXB13 (UniProt Q92826) reveals highly conserved amino acid domains (avg. pLDDT $\sim 64.1$) whose structural rigidity provides the necessary binding affinity landscapes to generate the segmented information profiles used in our model.

\subsection{Biological metric and effective energy}

The central hypothesis of the IEC framework is that developmental information modifies the \emph{effective} metric experienced by the rod's reference manifold. We define a \emph{biological metric} $d\ell_{\mathrm{eff}}^2$ that encodes target geometry via a local dilation factor $g_{\mathrm{eff}}(s)$:

\begin{equation}
d\ell_{\mathrm{eff}}^2 = g_{\mathrm{eff}}(s)\,ds^2 = \exp\left[2\left(\beta_1 \tilde{I}(s) + \beta_2 \frac{\partial \tilde{I}}{\partial s}\right)\right] ds^2,
\label{eq:biological_metric}
\end{equation}
where $\tilde{I}$ is the normalized information field and $\beta_{1,2}$ are dimensionless coupling constants. 

\paragraph{Justification of the functional form:} The exponential form is chosen for several physical and geometric reasons. First, it ensures $g_{\mathrm{eff}} > 0$ for all field configurations, preserving the metric's positivity. Second, it follows the convention of conformal rescaling in differential geometry ($g_{ab} \to e^{2\phi} g_{ab}$), where the scalar field $\phi = \beta_1 \tilde{I} + \beta_2 \partial_s \tilde{I}$ acts as a local conformal factor. In the weak-coupling limit ($\beta_{1,2} \ll 1$), we recover a linear perturbation $g_{\mathrm{eff}} \approx 1 + 2\beta_1 \tilde{I} + 2\beta_2 \partial_s \tilde{I}$, while the full form allows for the large geometric distortions required to model lordotic counter-curvature. We emphasize that this metric is a phenomenological ansatz used to interpret the resulting geometry; the forward mechanical model implements these effects through information-dependent rest states as derived below.

The mechanics of the biological rod are governed by the minimization of a total potential energy $\mathcal{E}_{\mathrm{total}}$ that couples bending, torsion, and extension to the information field and gravity. For a Cosserat rod with centerline $\mathbf{r}(s)$, the full energy functional is:

\begin{equation}
\mathcal{E}_{\mathrm{total}} = \int_0^L \left[ \frac{1}{2} B_{\mathrm{eff}}(\kappa - \kappa_{\mathrm{rest}})^2 + \frac{1}{2} C_{\mathrm{eff}} \tau^2 + \frac{1}{2} K_{\mathrm{eff}} \varepsilon^2 + \rho A \mathbf{r} \cdot \mathbf{g} \right] ds,
\label{eq:iec_energy}
\end{equation}
where: 
\begin{itemize}
    \item $B_{\mathrm{eff}}(s) = E_0 I_{\mathrm{area}} (1 + \chi_E I(s))$ is the information-modulated bending stiffness.
    \item $C_{\mathrm{eff}}(s) = G_0 J (1 + \chi_C I(s))$ is the torsional stiffness ($G_0$: shear modulus).
    \item $K_{\mathrm{eff}}(s) = E_0 A (1 + \chi_M I(s))$ is the extensional stiffness.
    \item $\kappa_{\mathrm{rest}}(s) = \kappa_0 + \chi_\kappa \partial_s I$ is the information-driven target curvature.
    \item $\tau$ and $\varepsilon$ are the local torsion and strain, respectively.
\end{itemize}
The weighting $w(I) = (1 + \chi_E I(s))$ penalizes deviations from the information-prescribed counter-curvature more heavily in high-identity regions (e.g., cervical/lumbar junctions).

\subsection{Cosserat force and moment balance}

The equilibrium configuration is found by finding the stationary state of the total potential energy. For a static rod subject to gravity $\mathbf{f}_g = \rho A \mathbf{g}$ and active moments induced by the information field gradients, the balance equations for internal force $\mathbf{n}$ and moment $\mathbf{m}$ are:

\begin{align}
\mathbf{n}'(s) + \mathbf{f}_g &= \mathbf{0}, \nonumber \\
\mathbf{m}'(s) + \mathbf{r}'(s) \times \mathbf{n}(s) + \mathbf{m}_{\mathrm{info}}'(s) &= \mathbf{0},
\label{eq:cosserat_balance}
\end{align}
where $\mathbf{m}_{\mathrm{info}}$ represents the active couple.

\paragraph{Boundary Conditions:} To model the human spine, we impose clamped-free boundary conditions. The sacral base ($s=0$) is rigidly fixed, while the cranial tip ($s=L$) is free of external loads:
\begin{align}
\mathbf{r}(0) = \mathbf{0}, \quad &\mathbf{d}_i(0) = \mathbf{e}_i \quad (\text{Clamped Base}) \nonumber \\
\mathbf{n}(L) = \mathbf{0}, \quad &\mathbf{m}(L) = \mathbf{0} \quad (\text{Free Tip})
\label{eq:boundary_conditions}
\end{align}
where $\{\mathbf{e}_i\}$ is the laboratory frame.

\subsection{Mode selection and spinal geometry}

The transition from a sagging C-shape to a counter-curved S-shape can be analyzed as a spectral shift in the rod's eigenmodes. In the linearized limit, small transverse deflections $y(s)$ satisfy:

\begin{equation}
\mathcal{L}_{\mathrm{IEC}}[y(s)] = \frac{d^2}{ds^2} \left( B_{\mathrm{eff}}(s) \frac{d^2 y}{ds^2} \right) - \frac{d}{ds} \left( N(s) \frac{dy}{ds} \right) = \lambda_n y_n(s),
\label{eq:mode_selection}
\end{equation}
where $N(s) = \int_s^L \rho A g \, ds'$ is the axial tension due to gravity. The information field modifies the stiffness landscape $B_{\mathrm{eff}}(s)$. For a uniform beam ($\chi_E = 0$), the lowest eigenvalue $\lambda_0$ corresponds to a monotonic C-shaped mode. As $\chi_\kappa$ and $\chi_E$ increase, the effective stiffness at the peaks of $I(s)$ (cervical and lumbar regions) increases, making the C-mode energetically unfavorable. At a critical coupling strength $\chi_{\mathrm{crit}}$, a \emph{mode crossing} occurs where the S-shaped mode (characterized by a second zero-crossing in curvature) becomes the ground state ($\lambda_0$).

\subsection{Geodesic curvature deviation metric}

To quantify the degree of biological countercurvature—the extent to which information reshapes equilibrium geometry beyond passive gravitational response—we define a normalized geodesic curvature deviation $\widehat{D}_{\mathrm{geo}}$. This metric compares the realized curvature profile $\kappa_{\mathrm{IEC}}(s)$ from the full IEC-coupled simulation to the reference passive curvature $\kappa_{\mathrm{passive}}(s)$ obtained with identical boundary conditions and loading but $\chi_\kappa = \chi_E = 0$:

\begin{equation}
D_{\mathrm{geo}} = \int_0^L \left| \kappa_{\mathrm{IEC}}(s) - \kappa_{\mathrm{passive}}(s) \right|^2 w(s)\, ds,
\label{eq:dgeo_raw}
\end{equation}
where $w(s) = g_{\mathrm{eff}}(s)$ is an optional weighting function reflecting the biological metric. For regime classification, we normalize by the maximum deviation observed across the parameter space:

\begin{equation}
\widehat{D}_{\mathrm{geo}} = \frac{D_{\mathrm{geo}}}{D_{\mathrm{geo,max}}(\chi_\kappa, g)},
\label{eq:dgeo_normalized}
\end{equation}
where $D_{\mathrm{geo,max}}$ is computed over the explored $(\chi_\kappa, g)$ domain. Physically, $\widehat{D}_{\mathrm{geo}} \approx 0$ indicates the system follows gravitational geodesics (passive regime), while $\widehat{D}_{\mathrm{geo}} \sim 1$ indicates strong information-driven geometric distortion. We classify regimes as: \emph{gravity-dominated} ($\widehat{D}_{\mathrm{geo}} < 0.1$), \emph{cooperative} ($0.1 \leq \widehat{D}_{\mathrm{geo}} \leq 0.3$), and \emph{information-dominated} ($\widehat{D}_{\mathrm{geo}} > 0.3$), with thresholds chosen to separate qualitatively distinct curvature morphologies observed in simulations.
