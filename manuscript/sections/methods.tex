\section{Methods}

\subsection{Information--elasticity coupling and beam model}

On a body axis $s\in[0,L]$, an information field $I(s)$ modulates rest curvature $\kappa_{\mathrm{rest}}(s)$, effective stiffness $E_{\mathrm{eff}}(s)$, and active moments $M_{\mathrm{info}}(s)$ through dimensionless couplings $\chi_{\kappa}$ (curvature), $\chi_{E}$ (stiffness), $\chi_{C}$ (cross-linking, analogous to nonlocal sliding parameters $\mu$ and $\gamma$ in counterbend mechanics), and $\chi_{M}$ (active moments). The information field $I(s)$ represents spatial patterns of biological activity---developmental gene expression gradients (e.g., HOX/PAX patterning that establishes segmental identity~\cite{wellik2007hox}), neural control signals, or morphogen distributions. A canonical spinal information field peaks in lumbar and cervical regions, corresponding to regions of high neural activity, and $\kappa_{\mathrm{gen}}(s)$ matches typical human sagittal curvature. A static cantilever beam under gravity provides the baseline equilibrium; information-coupled properties modify the solution relative to the passive (gravity-only) case.

The IEC coupling relations are:
\begin{align}
\kappa_{\mathrm{rest}}(s) &= \kappa_{0}(s) + \chi_{\kappa} \frac{\partial I(s)}{\partial s}, \label{eq:iec_curvature} \\
E_{\mathrm{eff}}(s) &= E_{0} \left(1 + \chi_{E} I(s)\right), \label{eq:iec_stiffness} \\
M_{\mathrm{info}}(s) &= \chi_{M} I(s) \kappa_{\mathrm{gen}}(s), \label{eq:iec_moment}
\end{align}
where $\kappa_{0}(s)$ is the gravity-selected baseline curvature and $E_{0}$ is the baseline stiffness.

\paragraph{Validation.} We validate the boundary-value solver against analytical Euler--Bernoulli sinusoidal solutions under the small-slope regime. With $L=1$\,m and two spatial periods, the numerical centerline matches the analytic profile to machine precision in double-precision arithmetic (L2 error $\lesssim 10^{-3}$ on coarse grids; boundary-condition residuals $\ll 10^{-12}$). Mesh refinement yields the expected decrease in error, confirming consistency of the discretization.

\subsection{Cosserat rod formulation and PyElastica implementation}

We promote the IEC beam to a full three-dimensional Cosserat rod implemented in PyElastica~\cite{pyelastica_zenodo,gazzola2018forward,antman2005nonlinear}, accounting for bending, twisting, and stretching with director frames along the rod. The rod is discretized into $n$ elements (typically $n=100$ for full-resolution simulations, $n=50$ for quick sweeps) with element-wise rest curvature, stiffness, and material properties. A clamped base and free end set boundary conditions, with gravity applied as a body force. Information-coupled properties are interpolated to elements. Time integration uses a Position-Verlet scheme with damping (damping coefficient $\gamma \sim 0.1$--$1.0$) to approach static equilibrium. For each experiment, we run the simulation until the maximum velocity falls below a threshold ($<10^{-6}$ m/s), ensuring convergence to static equilibrium.

The Cosserat rod equilibrium equations, incorporating IEC coupling, are:
\begin{align}
\frac{\partial \bm{F}}{\partial s} + \bm{f}_{\mathrm{ext}} &= \rho A \frac{\partial^{2}\bm{r}}{\partial t^{2}}, \label{eq:cosserat_force} \\
\frac{\partial \bm{M}}{\partial s} + \frac{\partial \bm{r}}{\partial s} \times \bm{F} + \bm{m}_{\mathrm{ext}} + M_{\mathrm{info}}(s) \hat{\bm{e}}_{3} &= \rho I \frac{\partial^{2}\bm{\theta}}{\partial t^{2}}, \label{eq:cosserat_moment}
\end{align}
where $\bm{F}$ and $\bm{M}$ are the internal force and moment vectors, $\bm{f}_{\mathrm{ext}}$ and $\bm{m}_{\mathrm{ext}}$ are external forces and moments (including gravity), $\bm{r}(s)$ is the centerline position, $\bm{\theta}(s)$ is the rotation vector, and $M_{\mathrm{info}}(s)$ is the information-driven active moment from Eq.~\eqref{eq:iec_moment}.

\subsection{Biological countercurvature metric and geodesic curvature deviation}

On $s\in[0,L]$, we define the biological countercurvature metric:
\begin{equation}
    d\ell_{\mathrm{eff}}^{2} = g_{\mathrm{eff}}(s)\,ds^{2}, \label{eq:metric}
\end{equation}
with the conformal factor:
\begin{equation}
    g_{\mathrm{eff}}(s) = \exp\bigl[2\phi(s)\bigr], \qquad
    \phi(s) = \beta_{1}\,\widetilde{I}_{\mathrm{centered}}(s) + \beta_{2}\,\widetilde{I}'(s), \label{eq:metric_factor}
\end{equation}
where $\widetilde{I}\in[0,1]$ is the normalized information field, $\widetilde{I}_{\mathrm{centered}} = \widetilde{I} - \langle\widetilde{I}\rangle$ (with $\langle\cdot\rangle$ denoting spatial average), $\widetilde{I}'$ is the normalized gradient, and $\beta_{1},\beta_{2}>0$ are dimensionless weights (typically $\beta_{1}=1.0$, $\beta_{2}=0.5$)~\cite{lee2018riemannian}. 

The geodesic curvature deviation between passive and information-coupled curvature profiles is:
\begin{equation}
    D_{\mathrm{geo}}^{2}
    = \int_{0}^{L} g_{\mathrm{eff}}(s)\,\bigl[\kappa_{I}(s)-\kappa_{0}(s)\bigr]^{2}\,ds, \label{eq:d_geo}
\end{equation}
with $\kappa_{0}$ the gravity-only curvature and $\kappa_{I}$ the information-coupled curvature. The normalized form is:
\begin{equation}
    \widehat{D}_{\mathrm{geo}}
    = \frac{D_{\mathrm{geo}}}{\sqrt{\int_{0}^{L} g_{\mathrm{eff}}(s)\,\kappa_{0}(s)^{2}\,ds} + \varepsilon}, \label{eq:d_geo_hat}
\end{equation}
with small regularization $\varepsilon \sim 10^{-6}$ to prevent division by zero. The metric $g_{\mathrm{eff}}(s)$ weights regions of high information density or gradient; $\widehat{D}_{\mathrm{geo}}$ quantifies departures from gravity-selected geodesics in this countercurvature geometry.

\begin{mathbox}[Biological Countercurvature Metric]
The biological metric $g_{\mathrm{eff}}(s)$ encodes how information processing reshapes effective geometry. Regions with high $I(s)$ or sharp gradients $\partial I/\partial s$ are assigned larger metric weight, effectively ``stretching'' the local coordinate system. This implements biological countercurvature as a phenomenological Riemannian metric on the rod's arc-length coordinate, analogous to how matter curves spacetime in general relativity~\cite{einstein1916grundlage,wald1984gr}.
\end{mathbox}

\subsection{Connection to counterbend mechanics}

The IEC framework connects to counterbend mechanics through the cross-linking parameter $\chi_{C}$, which is analogous to the nonlocal sliding parameters $\mu$ (sliding resistance) and $\gamma$ (basal compliance) in counterbend models. The relationship is:
\begin{equation}
\chi_{C} \propto \frac{\mu}{\gamma}, \label{eq:counterbend_analogy}
\end{equation}
where higher $\chi_{C}$ corresponds to increased nonlocal coupling between rod segments, similar to how cross-linked filaments in counterbend systems exhibit nonlocal sliding behavior. This connection provides a mechanistic bridge between developmental information fields and the mechanical properties of biological structures.

\subsection{Numerical experiments}

\paragraph{Spine-like S-curves.}
A spinal information field $I(s)$ peaks in lumbar and cervical regions; $\kappa_{\mathrm{gen}}(s)$ matches typical human sagittal curvature. We sweep $\chi_{\kappa}\in[0,0.05]$ at $g=9.81\,\mathrm{m/s}^{2}$ with clamped base and free tip.

\paragraph{Plant-like upward growth.}
A passive, sagging rod is compared to an information-driven rod that bends upward. The coupling $\chi_{\kappa}$ controls the transition; $\widehat{D}_{\mathrm{geo}}<0.1$ indicates gravity-dominated sag, while $\widehat{D}_{\mathrm{geo}}>0.2$ indicates strong information-driven upward bending.

\paragraph{Microgravity adaptation.}
A fixed information field $I(s)$ is run across $g\in\{1.0, 0.5, 0.1, 0.05, 0.01\}$ (in units of Earth gravity). Passive curvature energy and $\widehat{D}_{\mathrm{geo}}$ are computed to show information-driven structure persisting as gravity is reduced~\cite{green2018spinal,marfia2023microgravity}.

\subsection{Thoracic asymmetry and scoliosis metrics}

A localized thoracic bump on normalized $\hat{s}\in[0,1]$ is defined as:
\begin{equation}
    G(\hat{s}) = \exp\!\Bigl[-\tfrac{1}{2}\bigl((\hat{s}-\hat{s}_{0})/\sigma\bigr)^{2}\Bigr] \label{eq:asymmetry_bump}
\end{equation}
with $\hat{s}_{0}\approx0.6$ and width $\sigma$ spanning $\sim2$--$3$ vertebral levels. An otherwise symmetric field $I_{\mathrm{sym}}$ is perturbed by:
\begin{equation}
    I_{\mathrm{asym}}(s) = I_{\mathrm{sym}}(s) + \varepsilon_{\mathrm{asym}}\,\Delta I\,G(\hat{s}), \label{eq:asymmetric_field}
\end{equation}
with $\Delta I = \max I_{\mathrm{sym}} - \min I_{\mathrm{sym}}$ and $\varepsilon_{\mathrm{asym}}\sim0.05$. Alternatively, a lateral curvature bump $\kappa_{\mathrm{lat}}(s) = \varepsilon_{\mathrm{lat}}G(\hat{s})$ with $\varepsilon_{\mathrm{lat}} = 0.01$--$0.05\,\mathrm{m}^{-1}$ is added to the coronal rest curvature.

From coronal centerlines $(z(s),y(s))$, we define the lateral deviation index:
\begin{equation}
S_{\mathrm{lat}} = \frac{\max_{s}|y(s)|}{L_{\mathrm{eff}}}, \label{eq:slat}
\end{equation}
where $L_{\mathrm{eff}} = \max z - \min z$ is the effective length, and a Cobb-like angle $\theta_{\mathrm{Cobb}}$ from line fits to the lowest and highest 20\% of points. These metrics are evaluated for symmetric and asymmetric runs across $(\chi_{\kappa},g)$ and combined with $\widehat{D}_{\mathrm{geo}}$ to classify gravity-dominated, cooperative, and scoliosis-like regimes~\cite{weinstein2008adolescent,white_panjabi_spine}.
