\section{Methods: Deterministic Beam Model and PyElastica Simulations}

We implement the Information--Cosserat framework using two complementary numerical approaches: a fast deterministic beam model for parameter sweeps and eigenanalysis, and a full three-dimensional Cosserat rod simulation for capturing large deformations and geometric nonlinearities.

\subsection{Deterministic IEC Beam Model}

To explore the mode selection mechanism (Eq.~\ref{eq:mode_selection}), we discretize the linearized beam equations using a finite difference scheme on a 1D domain $s \in [0, L]$. The rod is divided into $N=100$ segments. The information field $I(s)$ is mapped to local stiffness $E_i$ and rest curvature $\kappa_{i}$ at each node. We solve the resulting boundary value problem (BVP) using a standard shooting method (or sparse matrix solver for the eigenproblem). This allows rapid exploration of the $(\chi_\kappa, g)$ parameter space to identify regions where S-modes become the ground state.

\subsection{3D Cosserat Rod Implementation (PyElastica)}

For full 3D simulations, we utilize \texttt{PyElastica}~\cite{pyelastica_zenodo,gazzola2018forward}, an open-source Python implementation of Cosserat rod theory. Model parameters are listed in Table~\ref{tab:parameters}. The spine is modeled as a Cosserat rod with the following specifications:

\begin{table}[h!]
\centering
\caption{Computational model parameters and biological justification}
\label{tab:parameters}
\small
\begin{tabular}{llll}
\toprule
\textbf{Parameter} & \textbf{Value} & \textbf{Units} & \textbf{Biological Basis} \\
\midrule
\multicolumn{4}{l}{\textit{Geometry}} \\
Length $L$ & 0.40 & m & Adult spine (sacrum to cranium) \\
Diameter $d$ & 0.03 & m & Typical vertebral body dimension \\
$n$ elements & 50--100 & -- & Convergence-tested discretization \\
\midrule
\multicolumn{4}{l}{\textit{Material Properties}} \\
Young's modulus $E_0$ & 1.0 & GPa & Effective modulus (bone + disc) \\
Area moment $I_{\mathrm{area}}$ & $1.0 \times 10^{-8}$ & m$^4$ & Cylindrical cross-section \\
Density $\rho$ & 1100 & kg/m$^3$ & Tissue-averaged density \\
Cross-section $A$ & $7.1 \times 10^{-4}$ & m$^2$ & $\pi (d/2)^2$ \\
\midrule
\multicolumn{4}{l}{\textit{IEC Coupling Parameters}} \\
$\chi_\kappa$ & 0--0.10 & m$^{-1}$ & Rest curvature coupling (sweep) \\
$\chi_E$ & 0.10 & -- & Stiffness modulation (fixed) \\
$\chi_M$ & 0--0.05 & N$\cdot$m & Active moment coupling \\
$\beta_1, \beta_2$ & 0.1, 0.05 & -- & Metric coupling constants \\
\midrule
\multicolumn{4}{l}{\textit{Information Field (Spinal Profile)}} \\
Cervical peak & $s/L = 0.80$ & -- & C1-C7 region \\
Lumbar peak & $s/L = 0.25$ & -- & L1-L5 region \\
Peak amplitude & 0.5--0.7 & -- & Normalized HOX expression \\
Baseline $I_0$ & 0.3 & -- & Background patterning \\
\midrule
\multicolumn{4}{l}{\textit{Loading \& Dynamics}} \\
Gravity $g$ & 0.01--1.0 & $g_\oplus$ & Microgravity to Earth \\
Damping $\nu$ & 0.1--1.0 & s$^{-1}$ & Numerical dissipation \\
Equilibrium criterion & $v_{\max} < 10^{-6}$ & m/s & Kinetic energy threshold \\
\bottomrule
\end{tabular}
\end{table}

\begin{itemize}
    \item \textbf{Discretization}: The rod is discretized into $n=50$--$100$ elements.
    \item \textbf{Information Field}: For spinal simulations, $I(s)$ is specified as a bimodal Gaussian distribution representing HOX-patterned identity:
    \begin{equation}
    I(s) = A_c \exp\left[-\frac{(s/L - s_c)^2}{2\sigma_c^2}\right] + A_l \exp\left[-\frac{(s/L - s_l)^2}{2\sigma_l^2}\right] + I_0,
    \label{eq:info_field_spinal}
    \end{equation}
    where $A_c = 0.5$, $s_c = 0.80$, $\sigma_c = 0.08$ (cervical lordosis), $A_l = 0.7$, $s_l = 0.25$, $\sigma_l = 0.10$ (lumbar lordosis), and $I_0 = 0.3$ (baseline). For scoliosis perturbations, a lateral asymmetry is added: $I(s) \to I(s) + \varepsilon_{\mathrm{asym}} \exp[-(s/L - 0.6)^2/(2 \cdot 0.08^2)]$ with $\varepsilon_{\mathrm{asym}} = 0.01$--$0.05$.
    \item \textbf{IEC Coupling}: We implemented a custom callback in \texttt{PyElastica} that updates the local rest curvature vector $\bm{\kappa}^0(s)$ and bending stiffness matrix $\mathbf{B}(s)$ at each time step (or initialization) based on the information field $I(s)$.
    \item \textbf{Boundary Conditions}: The rod is clamped at the base (sacrum) and free at the top (cranium), simulating a cantilever column under gravity. For specific validation cases, clamped-clamped conditions are used.
    \item \textbf{Gravitational Loading}: Gravity is applied as a uniform body force $\mathbf{f} = \rho A \mathbf{g}$.
    \item \textbf{Damping}: To find static equilibrium configurations, we apply external damping ($\nu \sim 0.1$--$1.0$ s$^{-1}$) and integrate the dynamic equations until the kinetic energy dissipates ($v_{\max} < 10^{-6}$ m/s).
\end{itemize}

The source code for the IEC-modified Cosserat solver is available in the \texttt{spinalmodes} Python package (see Data Availability).

\subsection{Parameter Sweeps and Mode Classification}

We perform systematic parameter sweeps over the coupling strength $\chi_\kappa$ (range $[0, 0.1]$) and gravitational acceleration $g$ (range $[0.01, 1.0]$ $g_{\mathrm{Earth}}$). For each simulation, we compute the equilibrium shape and evaluate the following metrics:
1.  \textbf{Geodesic Deviation} $\widehat{D}_{\mathrm{geo}}$: Quantifies the difference between the realized shape and the gravity-only geodesic.
2.  \textbf{Lateral Deviation} $S_{\mathrm{lat}}$: Measures symmetry breaking in the coronal plane.
3.  \textbf{Cobb Angle}: Standard clinical measure for scoliotic curves.

Regimes are classified based on the normalized geodesic deviation $\widehat{D}_{\mathrm{geo}}$. We define the \emph{gravity-dominated} regime ($\widehat{D}_{\mathrm{geo}} < 0.1$) as the region where gravitational potential energy dominates, leading to monotonic sag. The \emph{cooperative} regime ($0.1 \leq \widehat{D}_{\mathrm{geo}} \leq 0.3$) corresponds to the range where information and gravity balance to produce a stable, counter-curved S-shape. The \emph{information-dominated} regime ($\widehat{D}_{\mathrm{geo}} > 0.3$) is reached when information-driven metric distortions become the primary determinant of geometry, which, as we show, also coincides with the onset of symmetry-breaking instabilities (scoliosis-like modes). These thresholds were determined by analyzing the bifurcation of the lowest eigenvalue $\lambda_0$ and the emergence of non-monotonic curvature profiles.

\subsection{Numerical convergence and validation}

A convergence study was performed by varying the number of elements $n$ from $10$ to $200$. We found that the normalized geodesic deviation $\widehat{D}_{\mathrm{geo}}$ converges to within 1\% for $n \geq 50$ (see Supplementary Fig.~S1). The numerical implementation was validated against analytical solutions for small-deflection Euler-Bernoulli beams. The \texttt{PyElastica} implementation was further verified by reproducing standard buckling and hanging chain benchmarks, with error metrics $||\mathbf{r}_{num} - \mathbf{r}_{ana}||/L < 10^{-4}$.
