\section{Introduction}

\subsection{The puzzle of spinal curvature under gravity}
Living systems do not simply obey gravity; they negotiate with it. While a passive elastic beam clamped at one end and subject to gravity will sag into a monotonic C-shape, biological structures such as plant stems and vertebrate spines adopt complex, robust geometries that defy this passive tendency. The human spine, in particular, maintains a characteristic S-shaped sagittal profile (cervical and lumbar lordosis, thoracic kyphosis) that is critical for bipedal posture and shock absorption~\cite{white_panjabi_spine}. This shape is not merely a reaction to load but an intrinsic, actively maintained ``counter-curvature.''

\subsection{Developmental genetic patterning}
The blueprint for this geometry is laid down during embryogenesis. The paraxial mesoderm segments into somites, driven by the segmentation clock and oscillating gene expression (e.g., Notch, Wnt, FGF)~\cite{pourquie2011vertebrate}. These segments acquire distinct identities through the expression of HOX and PAX genes, which specify the morphological characteristics of the resulting vertebrae~\cite{wellik2007hox}. However, the mechanism by which these discrete genetic codes are translated into the continuous, mesoscale geometry of the adult spine remains a fundamental open question.

\subsection{Prior biomechanical models and their limitations}
Classical spinal biomechanics treats the spine as a passive elastic structure: either a continuous Euler-Bernoulli beam or a multi-body linkage of rigid vertebrae connected by compliant discs~\cite{white_panjabi_spine}. While these models successfully capture static load distribution and predict failure modes under extreme loading, they cannot explain several key observations:
\begin{enumerate}[label=(\roman*)]
\item \textbf{Robustness across environments}: The characteristic S-curve is maintained across a wide range of gravitational loads, from parabolic flight (transient microgravity) to long-duration spaceflight~\cite{green2018spinal}.
\item \textbf{Developmental stability}: The spinal curvature emerges reliably during development despite variations in mechanical loading (e.g., differences in prenatal activity or postnatal weight-bearing).
\item \textbf{Idiopathic pathology}: Adolescent idiopathic scoliosis (AIS) arises without detectable structural asymmetries in vertebrae or discs~\cite{weinstein2008adolescent}, suggesting a control or patterning defect rather than a purely mechanical failure.
\end{enumerate}
Our IEC framework addresses these gaps by incorporating developmental information as an \emph{active geometric source} that modifies the effective metric experienced by the spine, rather than treating curvature as a passive reaction to external loads.

\subsection{Hypothesis: Information--Elasticity Coupling and effective metric}
We propose that developmental information acts as a field that modifies the effective geometry experienced by the spine. Drawing an analogy to General Relativity, where matter curves spacetime, we suggest that biological information curves the ``material manifold'' of the spine. We formalize this as an \emph{Information--Elasticity Coupling (IEC)} framework, where a genetic information field $I(s)$ modifies the rest curvature, stiffness, and active moments of the structure. In this view, the spine does not ``fight'' gravity; it settles into the geodesic of a curved biological metric $d\ell_{\mathrm{eff}}^2$ shaped by information.

\subsection{Contribution and overview}
In this work, we:
(i) Define the IEC model and a phenomenological ``biological metric'' that maps genetic information to geometric distortions.
(ii) Implement this framework in a 3D Cosserat rod simulation (using \texttt{PyElastica}) to model the spine under gravitational loading.
(iii) Demonstrate that the interplay between gravity and information selects specific ``spinal modes''---shifting the ground state from a passive C-shape to an active S-shape.
(iv) Show that in information-dominated regimes, this same mechanism can amplify small asymmetries into pathological, scoliosis-like deformities.
