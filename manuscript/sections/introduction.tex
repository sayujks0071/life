\section{Introduction}

Living systems do not simply obey gravity; they negotiate with it. Human spines adopt robust S-shaped profiles~\cite{white_panjabi_spine}, plant stems grow upward, and neural structures adapt under microgravity~\cite{green2018spinal,marfia2023microgravity}. These behaviors suggest that biological information---developmental patterning, neural control, or genetic programs---actively reshapes the equilibrium geometry that gravity alone would select.

We frame this behavior as \emph{biological countercurvature}: information-driven modification of the effective geometry experienced by a body in a gravitational field. We couple an IEC model to three-dimensional Cosserat rod mechanics~\cite{antman2005nonlinear}, treating the rod in gravity as an analog spacetime and the IEC information field as a source of effective countercurvature. This perspective yields: (1) a biological metric $d\ell_{\mathrm{eff}}^{2} = g_{\mathrm{eff}}(s)\,ds^{2}$ derived from the information field $I(s)$ and its gradient; (2) a normalized geodesic curvature deviation $\widehat{D}_{\mathrm{geo}}$ quantifying how information reshapes equilibrium curvature relative to gravity-selected profiles; and (3) a phase diagram mapping gravity-dominated, cooperative, and information-dominated countercurvature regimes versus information-coupling strength and gravitational loading. Normal sagittal spinal curvature and scoliosis-like lateral deviations then emerge as different regimes of the same model, providing a quantitative link between information processing, mechanics, and geometry in living systems.
