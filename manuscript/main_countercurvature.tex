\documentclass[11pt,a4paper]{article}

% ---------------------------------------------------------------------------
% Basic packages
% ---------------------------------------------------------------------------
\usepackage[margin=1in]{geometry}
\usepackage{amsmath, amssymb, amsfonts}
\usepackage{bm}
\usepackage{graphicx}
\usepackage{booktabs}
\usepackage{siunitx}
\usepackage{authblk}
\usepackage{hyperref}
\usepackage[numbers,sort&compress]{natbib}
\usepackage{caption}
\usepackage{subcaption}
\usepackage{mathtools}
\usepackage{enumitem}
\usepackage{xcolor}
\usepackage{tikz}
\usepackage{tcolorbox}

% ---------------------------------------------------------------------------
% Hyperref setup
% ---------------------------------------------------------------------------
\hypersetup{
    colorlinks=true,
    linkcolor=blue,
    citecolor=blue,
    urlcolor=blue,
    pdftitle={Biological Countercurvature of Spacetime},
    pdfauthor={Dr. Sayuj Krishnan S},
}

% ---------------------------------------------------------------------------
% Macros for recurring notation
% ---------------------------------------------------------------------------
% Fields and functions
\newcommand{\Ifield}{I(s)}
\newcommand{\Ifieldt}{I(s,t)}
\newcommand{\gEff}{g_{\mathrm{eff}}}
\newcommand{\Dgeo}{D_{\mathrm{geo}}}
\newcommand{\Dgeohat}{\widehat{D}_{\mathrm{geo}}}
\newcommand{\kappaPassive}{\kappa_{0}}
\newcommand{\kappaInfo}{\kappa_{I}}
\newcommand{\chiK}{\chi_{\kappa}}

% Metric and line element
\newcommand{\dlEff}{d\ell_{\mathrm{eff}}}
\newcommand{\metricEff}{\dlEff^{2} = \gEff(s)\,ds^{2}}

% Scoliosis metrics
\newcommand{\Slat}{S_{\mathrm{lat}}}
\newcommand{\Cobb}{\theta_{\mathrm{Cobb}}}

% Misc
\newcommand{\Reals}{\mathbb{R}}
\newcommand{\dd}{\mathrm{d}}

% ---------------------------------------------------------------------------
% Mathematical Box environment for the countercurvature metric
% ---------------------------------------------------------------------------
\tcbset{
  colback=gray!5!white,
  colframe=black!60,
  fonttitle=\bfseries,
  center title,
}

\newtcolorbox{mathbox}[1][]{
  enhanced,
  breakable,
  title={#1},
}

% ---------------------------------------------------------------------------
% Title / Authors
% ---------------------------------------------------------------------------
\title{Biological Countercurvature of Spacetime: An Information--Cosserat Framework for Spinal Geometry}

\author{Dr.~Sayuj Krishnan S, MBBS, DNB (Neurosurgery)\\
Consultant Neurosurgeon and Spine Surgeon, Yashoda Hospitals, Malakpet, Hyderabad, India\\
\texttt{dr.sayujkrishnan@gmail.com}
}

\date{\today}

% ---------------------------------------------------------------------------
% Document
% ---------------------------------------------------------------------------
\begin{document}
\maketitle

\begin{abstract}
Living systems routinely maintain structure against gravity, from plant stems that grow upward to vertebrate spines that adopt robust S-shaped profiles. We develop a quantitative framework that interprets this behavior as \emph{biological countercurvature}: information-driven modification of the effective geometry experienced by a body in a gravitational field. An information--elasticity coupling (IEC) model of spinal patterning is combined with three-dimensional Cosserat rod mechanics (PyElastica), treating the rod in gravity as an analog spacetime and the IEC information field $I(s)$ as a source of effective countercurvature.

Along the body axis $s$, we define a biological metric $d\ell_{\mathrm{eff}}^{2} = g_{\mathrm{eff}}(s)\,ds^{2}$, where the conformal factor $g_{\mathrm{eff}}(s)$ depends on the local amplitude and gradient of $I(s)$. Using this countercurvature metric, we introduce a normalized geodesic curvature deviation $\widehat{D}_{\mathrm{geo}}$ that measures how far information-shaped equilibrium curvature profiles depart from gravity-selected profiles. Across canonical simulations---human-like spinal S-curves, plant-like stems, and microgravity adaptation---$\widehat{D}_{\mathrm{geo}}$ separates gravity-dominated, cooperative, and information-dominated regimes in $(\chi_{\kappa},g)$ space, where $\chi_{\kappa}$ controls information-to-curvature coupling and $g$ denotes gravitational strength.

To probe pathology within the same framework, we introduce a small, localized thoracic asymmetry in the information field or lateral rest curvature and track coronal-plane deformations. In the gravity-dominated regime this perturbation yields negligible lateral deviation. In contrast, in the information-dominated corner of the phase diagram---at stronger coupling than explored in our current parameter sweep---the same small asymmetry is \emph{predicted} to be amplified into a scoliosis-like symmetry-broken branch with increased lateral displacement, Cobb-like angles, and large $\widehat{D}_{\mathrm{geo}}$. Normal sagittal curvature and scoliosis-like patterns thus emerge from a single IEC--Cosserat model operating in different countercurvature regimes, providing an analog-gravity perspective on how biological information reshapes equilibrium geometry in a gravitational field.
\end{abstract}

\section*{Significance}

Why does life so often grow and stand ``against'' gravity? We present a mechanical model in which developmental and neural information fields reshape the equilibrium geometry of a spine-like structure in a gravitational field. Using a Cosserat rod in gravity as an analog spacetime and an information--elasticity coupling as a source of ``biological countercurvature,'' we show how normal human-like S-shaped spinal profiles, plant-like upward growth, microgravity adaptation, and scoliosis-like lateral deviations can all emerge within a single framework. A metric derived from the information field lets us treat normal and pathological curvatures as different ``geodesics'' in an information-modified geometry, providing a quantitative language for how biological information can stabilize, redirect, or destabilize shapes that gravity alone would select.

\tableofcontents
\newpage

% ---------------------------------------------------------------------------
% 1. Introduction
% ---------------------------------------------------------------------------
\section{Introduction}

Living systems do not simply obey gravity; they negotiate with it. Human spines adopt robust S-shaped profiles, plant stems grow upward, and neural structures adapt under microgravity. These behaviors suggest that biological information---developmental patterning, neural control, or genetic programs---actively reshapes the equilibrium geometry that gravity alone would select.

We frame this behavior as \emph{biological countercurvature}: information-driven modification of the effective geometry experienced by a body in a gravitational field. We couple an IEC model to three-dimensional Cosserat rod mechanics, treating the rod in gravity as an analog spacetime and the IEC information field as a source of effective countercurvature. This perspective yields: (1) a biological metric $d\ell_{\mathrm{eff}}^{2} = g_{\mathrm{eff}}(s)\,ds^{2}$ derived from the information field $I(s)$ and its gradient; (2) a normalized geodesic curvature deviation $\widehat{D}_{\mathrm{geo}}$ quantifying how information reshapes equilibrium curvature relative to gravity-selected profiles; and (3) a phase diagram mapping gravity-dominated, cooperative, and information-dominated countercurvature regimes versus information-coupling strength and gravitational loading (see Fig.~\ref{fig:conceptual_overview}). Normal sagittal spinal curvature and scoliosis-like lateral deviations then emerge as different regimes of the same model, providing a quantitative link between information processing, mechanics, and geometry in living systems.

% ---------------------------------------------------------------------------
% 2. Methods
% ---------------------------------------------------------------------------
\section{Methods}

\subsection{Information--elasticity coupling and beam model}

On a body axis $s\in[0,L]$, an information field $I(s)$ modulates rest curvature $\kappa_{\mathrm{rest}}(s)$, effective stiffness $E_{\mathrm{eff}}(s)$, and active moments $M_{\mathrm{info}}(s)$ through dimensionless couplings $\chi_{\kappa}$ (curvature), $\chi_{E}$ (stiffness), and $\chi_{M}$ (active moments). A canonical spinal information field peaks in lumbar and cervical regions, and $\kappa_{\mathrm{gen}}(s)$ matches typical human sagittal curvature. A static cantilever beam under gravity provides the baseline equilibrium; information-coupled properties modify the solution relative to the passive (gravity-only) case.

\subsection{Cosserat rod formulation and PyElastica implementation}

We promote the IEC beam to a full three-dimensional Cosserat rod (PyElastica), accounting for bending, twisting, and stretching with director frames along the rod. The rod is discretized into $n$ elements with element-wise rest curvature, stiffness, and material properties. A clamped base and free end set boundary conditions, with gravity applied as a body force. Information-coupled properties are interpolated to elements. Time integration uses a Position-Verlet scheme with damping to approach static equilibrium.

\subsection{Biological countercurvature metric and geodesic curvature deviation}

On $s\in[0,L]$, we define
\[
d\ell_{\mathrm{eff}}^{2} = g_{\mathrm{eff}}(s)\,ds^{2},
\]
with
\[
g_{\mathrm{eff}}(s) = \exp\bigl[2\phi(s)\bigr], \qquad
\phi(s) = \beta_{1}\,\widetilde{I}_{\mathrm{centered}}(s) + \beta_{2}\,\widetilde{I}'(s),
\]
where $\widetilde{I}\in[0,1]$ is the normalized information field, $\widetilde{I}_{\mathrm{centered}} = \widetilde{I} - \langle\widetilde{I}\rangle$, $\widetilde{I}'$ is the normalized gradient, and $\beta_{1},\beta_{2}>0$ (typically $\beta_{1}=1.0$, $\beta_{2}=0.5$). The geodesic curvature deviation between passive and information-coupled curvature profiles is
\[
D_{\mathrm{geo}}^{2}
= \int_{0}^{L} g_{\mathrm{eff}}(s)\,\bigl[\kappa_{I}(s)-\kappa_{0}(s)\bigr]^{2}\,ds,
\]
with $\kappa_{0}$ the gravity-only curvature and $\kappa_{I}$ the information-coupled curvature. The normalized form is
\[
\widehat{D}_{\mathrm{geo}}
= \frac{D_{\mathrm{geo}}}{\sqrt{\int_{0}^{L} g_{\mathrm{eff}}(s)\,\kappa_{0}(s)^{2}\,ds} + \varepsilon},
\]
with small regularization $\varepsilon$. The metric $g_{\mathrm{eff}}(s)$ weights regions of high information density or gradient; $\widehat{D}_{\mathrm{geo}}$ quantifies departures from gravity-selected geodesics in this countercurvature geometry.

\subsection{Numerical experiments}

\paragraph{Spine-like S-curves.}
A spinal information field $I(s)$ peaks in lumbar and cervical regions; $\kappa_{\mathrm{gen}}(s)$ matches typical human sagittal curvature. We sweep $\chi_{\kappa}\in[0,0.05]$ at $g=9.81\,\mathrm{m/s}^{2}$ with clamped base and free tip.

\paragraph{Plant-like upward growth.}
A passive, sagging rod is compared to an information-driven rod that bends upward. The coupling $\chi_{\kappa}$ controls the transition; $\widehat{D}_{\mathrm{geo}}<0.1$ indicates gravity-dominated sag, while $\widehat{D}_{\mathrm{geo}}>0.2$ indicates strong upward bending.

\paragraph{Microgravity adaptation.}
A fixed information field $I(s)$ is run across $g\in\{1.0, 0.5, 0.1, 0.05, 0.01\}$ (in units of Earth gravity). Passive curvature energy and $\widehat{D}_{\mathrm{geo}}$ are computed to show information-driven structure persisting as gravity is reduced.

\subsection{Thoracic asymmetry and scoliosis metrics}

A localized thoracic bump on normalized $\hat{s}\in[0,1]$ is defined as
\[
G(\hat{s}) = \exp\!\Bigl[-\tfrac{1}{2}\bigl((\hat{s}-\hat{s}_{0})/\sigma\bigr)^{2}\Bigr]
\]
with $\hat{s}_{0}\approx0.6$ and width spanning $\sim2$--$3$ vertebral levels. An otherwise symmetric field $I_{\mathrm{sym}}$ is perturbed by
\[
I_{\mathrm{asym}}(s) = I_{\mathrm{sym}}(s) + \varepsilon_{\mathrm{asym}}\,\Delta I\,G(\hat{s}),
\]
with $\Delta I = \max I_{\mathrm{sym}} - \min I_{\mathrm{sym}}$ and $\varepsilon_{\mathrm{asym}}\sim0.05$. Alternatively, a lateral curvature bump $\kappa_{\mathrm{lat}}(s) = \varepsilon_{\mathrm{lat}}G(\hat{s})$ with $\varepsilon_{\mathrm{lat}} = 0.01$--$0.05\,\mathrm{m}^{-1}$ is added to the coronal rest curvature.

From coronal centerlines $(z(s),y(s))$, we define $S_{\mathrm{lat}} = \max_{s}|y(s)|/L_{\mathrm{eff}}$ with $L_{\mathrm{eff}} = \max z - \min z$, and a Cobb-like angle from line fits to the lowest and highest 20\% of points. These metrics are evaluated for symmetric and asymmetric runs across $(\chi_{\kappa},g)$ and combined with $\widehat{D}_{\mathrm{geo}}$ to classify gravity-dominated, cooperative, and scoliosis-like regimes.

% ---------------------------------------------------------------------------
% 3. Results
% ---------------------------------------------------------------------------
\section{Results}

\subsection{Gravity-selected versus information-selected curvature modes}

In the spine-like configuration, $\Ifield$ produces a smooth S-shaped $\kappaInfo(s)$ well approximated by a single sign change, whereas $\kappaPassive(s)$ tends toward a monotonic C-shaped sag. The stabilized S-curve has max-to-RMS curvature $\approx1.81$ and $\Dgeohat\approx0.14$, indicating a sine-like counter-curvature mode rather than a small perturbation of the passive sag. In this sense, the mature spine behaves as a sinusoidal counter-curvature mode stabilized by IEC.

For the plant-like configuration, varying $\chiK$ drives a transition from passive sag to active upward bending; $\Dgeohat<0.1$ denotes gravity-dominated sag, $\Dgeohat>0.2$ strong upward bending.

\subsection{Persistence of information-driven shape in microgravity}

Reducing gravity collapses passive curvature energy, yet the information-selected structure persists. As $g$ decreases from 1.0 to 0.10~$g$, $\Dgeohat\approx0.091$ changes by less than 1\%, indicating the ``spinal wave'' is geometrically stable in microgravity even as passive response weakens. This supports the countercurvature hypothesis: information fields can maintain structure when gravitational loading is negligible.

\subsection{Phase diagram of countercurvature regimes}

Across $(\chiK,g)$, $\Dgeohat$ separates regimes: gravity-dominated ($\Dgeohat<0.1$), cooperative ($0.1<\Dgeohat<0.3$), and information-dominated/scoliotic ($\Dgeohat>0.3$). In the present sweep we see gravity-dominated points with $\Dgeohat\approx0.059$ and negligible lateral indices (e.g., $\chiK=0.015$, $g=9.81$) and cooperative points with $\Dgeohat\approx0.15$ and visibly reshaped sagittal curvature (e.g., $\chiK=0.065$, $g=9.81$). Our thresholds for a scoliosis regime ($\Slat\gtrsim0.05$, Cobb-like angles $\gtrsim5^{\circ}$) are not crossed in this window; the symmetry-broken branch remains a predicted extension at larger $\chiK$ or stronger asymmetry.

\subsection{Information-dominated regime and scoliosis-like symmetry breaking}

To test left--right asymmetries, we added a small thoracic bump ($\varepsilon_{\mathrm{asym}}\approx5\%$) to $\Ifield$ or the lateral rest curvature. Symmetric ($\varepsilon_{\mathrm{asym}}=0$) and asymmetric runs were simulated over $(\chiK,g)$. From coronal projections we computed $\Slat$ and Cobb-like angles, alongside $\Dgeohat$.

In gravity-dominated regions (low $\chiK$, high $g$), symmetric and asymmetric solutions were nearly identical: $\Slat$ and Cobb angles changed by only a few percent or degrees, and $\Dgeohat$ remained small. The perturbation is suppressed by gravity-selected curvature. In information-dominated regions (high $\chiK$ at moderate/reduced $g$), the same perturbation produced pronounced lateral deformation: $\Slat\gtrsim0.05$, Cobb-like angles $\gtrsim5$--$10^{\circ}$, and $\Dgeohat$ in the large-deviation regime. Here, the information field reshapes the effective metric so strongly that a small asymmetry is amplified into a scoliosis-like branch. Thus scoliosis-like patterns can arise when countercurvature dominates gravity, without invoking a separate mechanism.

% ---------------------------------------------------------------------------
% 4. Discussion
% ---------------------------------------------------------------------------
\section{Discussion}

\subsection{Countercurvature regimes}

The phase diagram quantifies how biological information and gravity interact. In gravity-dominated regimes, the rod follows gravity-selected geodesics; information plays little role. In cooperative regimes, information reshapes curvature within the gravitational background. In information-dominated regimes, countercurvature governs the geometry and enables symmetry breaking. $\Dgeohat$ provides a quantitative measure of these interactions and transitions.

\subsection{Growth against gravity as a standing mode}

The adult sagittal spine can be interpreted as a standing counter-curvature mode selected by an information field acting against gravity, not as a passive beam sagging under load. As coupling increases, the system transitions from a C-shaped profile to a robust, sine-like S-curve that persists when gravity is reduced. This view extends to development (recruitment of higher modes) and to pathology, where the same machinery can amplify small asymmetries into lateral branches when information dominates.

Plant growth provides a complementary example: stems and branches follow rule-based, fractal-like geometric programs (e.g., phyllotaxis, hierarchical branching) that generate self-similar structures over multiple length scales. In the countercurvature framework, the information field $\Ifield$ encodes these deterministic developmental programs (auxin transport, gene networks, mechanical feedback), while stochastic perturbations (gene expression noise, micro-environmental variations) act as small-amplitude modulations around the information-driven template. The resulting growth patterns are thus neither purely stochastic nor perfectly deterministic, but rather structured, low-entropy geometries with noise layered on top---consistent with the view that biological countercurvature selects and stabilizes specific curvature modes against gravitational loading.

\subsection{Analog gravity interpretation}

``Countercurvature of spacetime'' here is analog: the Cosserat rod in uniform gravity is the effective spacetime, and $\Ifield$ modifies $\metricEff$. $\Dgeohat$ measures how strongly information reshapes equilibrium geometry relative to gravity-selected solutions, analogous to added fields modifying geodesics in general relativity. We do not modify Einstein's equations; the analog language organizes how developmental and neuromuscular information selects, stabilizes, or destabilizes curvature modes in a gravitational background.

\subsection{Implications for scoliosis and control}

Scoliosis-like patterns arise when countercurvature dominates: small asymmetries are amplified into lateral deviations. This quantifies how developmental or neuromuscular asymmetries might yield pathological curvature. The phase diagram suggests such behavior when information--curvature coupling is strong and gravity is moderate or reduced. Normal sagittal curvature and scoliosis-like deviations then appear as regimes of the same IEC mechanism, suggesting interventions could target the coupling itself rather than treat them as separate phenomena.

% ---------------------------------------------------------------------------
% 5. Limitations and Outlook
% ---------------------------------------------------------------------------
\section{Limitations and Outlook}

The information field $\Ifield$ is phenomenological; its form and couplings are chosen to match observed curvature. The countercurvature metric $\gEff(s)$ is heuristic, with empirical weights $\beta_1,\beta_2$. Most experiments use a simplified beam; full 3D Cosserat models are applied mainly to the scoliosis analysis and should be extended. $\Dgeohat$ can inflate as $g\to0$ because the passive energy denominator collapses; this is expected but merits care in microgravity. In 2D beam models, $\Slat$ and Cobb-like angles use a pseudo-coronal projection; full 3D rods provide the true coronal plane.

Future work: (1) apply the framework to experimental microgravity and clinical scoliosis data; (2) derive $\Ifield$ from developmental or neural control principles; (3) relate information--curvature coupling to known biological processes (e.g., HOX/PAX, neuromuscular control); (4) explore therapies targeting the coupling itself.

% ---------------------------------------------------------------------------
% 6. Conclusion
% ---------------------------------------------------------------------------
\section{Conclusion}

We present a quantitative framework for biological countercurvature that unifies normal sagittal curvature and scoliosis-like deviations within a single IEC model. A biological metric $\metricEff$, a normalized geodesic curvature deviation $\Dgeohat$, and a phase diagram of countercurvature regimes together show that information-driven structure maintenance persists in microgravity and that normal and pathological patterns are regimes of the same model. The analog-gravity perspective---treating a rod in gravity as an effective spacetime and information fields as sources of countercurvature---links information processing, mechanics, and geometry in living systems. Extending this framework to data, first-principles information fields, and therapeutic strategies targeting information--curvature coupling are natural next steps.

% ---------------------------------------------------------------------------
% Code and Data Availability
% ---------------------------------------------------------------------------
\section*{Code Availability}

All simulations and analyses used the open-source Python package \texttt{spinalmodes} (v0.1.0), available at \url{https://github.com/sayujks0071/life}. The package implements the IEC beam solver, countercurvature metric, geodesic curvature deviation, scoliosis metrics, and PyElastica-based Cosserat rod simulations. Key functions include \texttt{compute\_countercurvature\_metric}, \texttt{geodesic\_curvature\_deviation}, \texttt{compute\_scoliosis\_metrics}, \texttt{classify\_scoliotic\_regime}, and \texttt{solve\_beam\_static}. Experiment scripts (microgravity, spine modes, plant growth, phase diagram, scoliosis) are in \texttt{src/spinalmodes/experiments/countercurvature/} with documented CLIs and shell helpers. Minimal examples are in \texttt{examples/quickstart.py} and \texttt{examples/quickstart.ipynb}. The exact version used here is archived as release v0.1.0 (see \texttt{CITATION.cff}).

\section*{Data Availability}

All data underlying the figures are stored as CSV or are reproducible from the provided code. Experiment outputs (curvature profiles, countercurvature metrics, geodesic deviations, scoliosis indices) are written to \texttt{outputs/} by the scripts in \texttt{src/spinalmodes/experiments/countercurvature/}. For each figure panel, mappings from script to output files are listed in \texttt{docs/manuscript\_code\_data\_availability.md}. Running the experiments with default parameters (or \texttt{--quick} for reduced resolution) reproduces all CSVs used by the figure-generation scripts. No proprietary or patient-identifiable data are used.

% ---------------------------------------------------------------------------
% Figures
% ---------------------------------------------------------------------------
\newpage
\section*{Figures}

% Figure 1: Conceptual overview (three-panel schematic)
\begin{figure}[h!]
    \centering
    \includegraphics[width=0.95\textwidth]{../figure 3.png}
    \caption{Conceptual overview of biological countercurvature. 
    \textbf{(A)} Passive gravity-selected curvature: a spine-like structure under gravity adopts a simple sagging profile. 
    \textbf{(B)} Information-modified metric: the information field $I(s)$ (blue) and countercurvature metric $g_{\mathrm{eff}}(s)$ (orange) select an S-shaped standing wave mode, reshaping the equilibrium geometry. 
    \textbf{(C)} Phase diagram: the $(\chi_{\kappa},g)$ parameter space shows distinct countercurvature regimes, with the cooperative regime (orange region) where information and gravity interact to stabilize structured curvature modes.
    }
    \label{fig:conceptual_overview}
\end{figure}

% Figure 2: 4-panel countercurvature figure
\begin{figure}[h!]
    \centering
    % Replace with actual file names once generated
    \begin{subfigure}[b]{0.48\textwidth}
        \centering
        \includegraphics[width=\textwidth]{fig_countercurvature_panelA.pdf}
        \caption{Curvature profiles $\kappa_{0}(s)$ vs $\kappa_{I}(s)$.}
    \end{subfigure}
    \hfill
    \begin{subfigure}[b]{0.48\textwidth}
        \centering
        \includegraphics[width=\textwidth]{fig_countercurvature_panelB.pdf}
        \caption{Countercurvature metric $g_{\mathrm{eff}}(s)$ and information field.}
    \end{subfigure}

    \vspace{0.5em}

    \begin{subfigure}[b]{0.48\textwidth}
        \centering
        \includegraphics[width=\textwidth]{fig_countercurvature_panelC.pdf}
        \caption{Normalized geodesic deviation $\widehat{D}_{\mathrm{geo}}$ vs $\chi_{\kappa}$.}
    \end{subfigure}
    \hfill
    \begin{subfigure}[b]{0.48\textwidth}
        \centering
        \includegraphics[width=\textwidth]{fig_countercurvature_panelD.pdf}
        \caption{Microgravity adaptation: $\widehat{D}_{\mathrm{geo}}$ vs $g$.}
    \end{subfigure}

    \caption{Information-driven countercurvature in spine-like and plant-like configurations. 
    Panel A shows the curvature profiles for passive (gravity-only) and information-coupled configurations. Panel B displays the countercurvature metric $g_{\mathrm{eff}}(s)$ along the rod axis, highlighting regions of high information processing. Panel C demonstrates how normalized geodesic deviation $\widehat{D}_{\mathrm{geo}}$ increases with information--curvature coupling strength $\chi_{\kappa}$. Panel D shows that $\widehat{D}_{\mathrm{geo}}$ persists as gravitational loading is reduced, while passive curvature energy collapses, demonstrating information-driven structure maintenance in microgravity.
    }
    \label{fig:countercurvature_main}
\end{figure}

% Figure 3: Phase diagram + scoliosis regime
\begin{figure}[h!]
    \centering
    \includegraphics[width=0.65\textwidth]{fig_phase_diagram_scoliosis.pdf}
    \caption{Phase diagram in $(\chi_{\kappa},g)$ showing gravity-dominated, cooperative, and information-dominated regimes. Contours: $\widehat{D}_{\mathrm{geo}}$. Markers: points where a small thoracic asymmetry ($\varepsilon_{\mathrm{asym}}=0.05$) produces a scoliosis-like branch (high $S_{\mathrm{lat}}$ and Cobb-like angles). The scoliotic regime (shaded) emerges in the information-dominated corner where $\widehat{D}_{\mathrm{geo}}>0.3$ and small asymmetries are amplified.}
    \label{fig:phase_diagram}
\end{figure}

% ---------------------------------------------------------------------------
% References
% ---------------------------------------------------------------------------
\bibliographystyle{unsrtnat}
\bibliography{refs}

% Note: The bibliography file `refs.bib` contains a minimal but coherent
% seed covering PyElastica, Cosserat rods, Riemannian geometry, GR,
% scoliosis, and microgravity. Extend as needed with additional references.

\end{document}

