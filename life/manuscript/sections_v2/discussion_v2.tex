\section{Discussion}

\subsection{Bridging the Genotype-Phenotype Gap}
The central achievement of this work is the establishment of a quantitative link between microscopic genetic patterning and macroscopic spinal geometry. Traditionally, these fields have been disconnected: developmental biologists study HOX expression domains, while biomechanists analyze beam buckling. Experiments using AlphaFold-derived parameters ($EI$, $\epsilon$, $g_{\mathrm{vol}}$) within our IEC framework demonstrate that these are not separate phenomena but coupled scales of the same physical problem. The "information field" $I(s)$ is the emergent order parameter that translates discrete molecular asymmetries (e.g., DNAH5 chirality) into the continuous manifold of the adult spine.

\subsection{Interpretation of Biological Countercurvature}
Our results suggest that the S-shape does not merely resist gravity but actively exploits it. The Information--Elasticity Coupling effectively "warps" the material metric, creating a potential well where the counter-curvature mode is the stable ground state. This explains the paradoxical observation that the spine maintains its shape in microgravity: the geometry is encoded in the metric itself ($d\ell_{\mathrm{eff}}^2$), not just in the reaction forces. 

\subsection{Scoliosis as a Decoding Error}
The "Scoliotic Canyon" observed in our phase diagrams offers a novel etiology for Idiopathic Scoliosis. Rather than a mechanical failure (buckling under load) or a purely genetic defect, scoliosis emerges as a \textit{decoding error}. When the coupling strength $\chi_\kappa$ is too high (information-dominated regime), the spine becomes hypersensitive to microscopic noise (e.g., small $\epsilon$ fluctuations). In this unstable regime, the system accesses high-order eigenmodes that break bilateral symmetry—manifesting clinically as the 3D helical deformity of scoliosis.

\subsection{Limits and Future Work}
While our model successfully integrates molecular and mechanical data, it assumes a static, effective information field. Real development is dynamic, growing in both size and complexity. Future work will extend this framework to 4D, coupling the IEC metric to volumetric growth laws. Additionally, patient-specific modeling using genomic data (e.g., inferring personal $\epsilon$ or $\chi_\kappa$ values from AlphaFold analysis of patient variants) could pave the way for predictive medicine in spinal deformities.
