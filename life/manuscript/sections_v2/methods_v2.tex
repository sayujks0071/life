\section{Methods}

We implement the Information--Cosserat framework using two complementary numerical approaches: a fast deterministic beam model for parameter sweeps and eigenanalysis, and a full three-dimensional Cosserat rod simulation for capturing large deformations and geometric nonlinearities.

\subsection{Parameter Derivation from Structural Biology}
To ground our model in biological reality, mechanical parameters were derived from AlphaFold structural predictions of key spinal proteins (see Supplementary Information).
\begin{enumerate}
    \item \textbf{Stiffness Distribution ($EI(s)$)}: Derived from the helical content and packing density of Collagen II (COL2A1, $EI_{\mathrm{norm}} \approx 0.90$) and Aggrecan (ACAN, $EI_{\mathrm{norm}} \approx 0.13$). The stiffness profile $B(s)$ mirrors the expression density of these ECM components.
    \item \textbf{Symmetry Breaking ($\epsilon$)}: The magnitude of the chiral torque $\tau_0$ is scaled by the molecular asymmetry of the ciliary dynein motor protein (DNAH5), yielding a normalized symmetry-breaking parameter $\epsilon \approx 0.95$ in the nodal region.
    \item \textbf{Volumetric Growth ($g_{\mathrm{vol}}$)}: The osmotic swelling potential of the intervertebral disc is parameterized by the intrinsic disorder and glycosylation sites of Aggrecan ($G_{\mathrm{vol}} \approx 1.60$).
\end{enumerate}

\subsection{Deterministic IEC Beam Model}
To explore the mode selection mechanism (Eq.~\ref{eq:mode_selection}), we discretize the linearized beam equations using a finite difference scheme on a 1D domain $s \in [0, L]$. The rod is divided into $N=100$ segments. The information field $I(s)$ is mapped to local stiffness $E_i$ and rest curvature $\kappa_{i}$ at each node. We solve the resulting boundary value problem (BVP) using a standard shooting method (or sparse matrix solver for the eigenproblem). This allows rapid exploration of the $(\chi_\kappa, g)$ parameter space to identify regions where S-modes become the ground state.

\subsection{3D Cosserat Rod Implementation (PyElastica)}
For full 3D simulations, we utilize \texttt{PyElastica}~\cite{pyelastica_zenodo,gazzola2018forward}, an open-source Python implementation of Cosserat rod theory. The rod is discretized into $n=50$--$100$ elements. We implemented a custom callback that updates the local rest curvature vector $\bm{\kappa}^0(s)$ and bending stiffness matrix $\mathbf{B}(s)$ at each time step based on the information field $I(s)$ and the derived structural parameters.

\textbf{Boundary Conditions}: The rod is clamped at the base (sacrum) and free at the top (cranium), simulating a cantilever column under gravity. Gravity is applied as a uniform body force $\mathbf{f} = \rho A \mathbf{g}$. To find static equilibrium configurations, we apply external damping ($\nu \sim 0.1$--$1.0$) and integrate until kinetic energy dissipates ($v_{\max} < 10^{-6}$ m/s).

\subsection{Parameter Sweeps and Mode Classification}
We perform systematic parameter sweeps over the coupling strength $\chi_\kappa$ (range $[0, 0.1]$) and gravitational acceleration $g$ (range $[0.01, 1.0]$ $g_{\mathrm{Earth}}$). For each simulation, we compute:
1.  \textbf{Geodesic Deviation} $\widehat{D}_{\mathrm{geo}}$: Difference between the realized shape and the gravity-only geodesic.
2.  \textbf{Lateral Deviation} $S_{\mathrm{lat}}$: Symmetry breaking in the coronal plane.
3.  \textbf{Cobb Angle}: Standard clinical measure for scoliotic curves.

Regimes are classified as \emph{gravity-dominated} ($\widehat{D}_{\mathrm{geo}} < 0.1$), \emph{cooperative} ($0.1 < \widehat{D}_{\mathrm{geo}} < 0.3$), or \emph{information-dominated} ($\widehat{D}_{\mathrm{geo}} > 0.3$).
