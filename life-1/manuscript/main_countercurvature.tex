\documentclass[11pt,a4paper]{article}

% ---------------------------------------------------------------------------
% Basic packages
% ---------------------------------------------------------------------------
\usepackage[margin=1in]{geometry}
\usepackage{amsmath, amssymb, amsfonts}
\usepackage{bm}
\usepackage{graphicx}
\usepackage{booktabs}
\usepackage{siunitx}
\usepackage{authblk}
\usepackage{hyperref}
\usepackage[numbers,sort&compress]{natbib}
\usepackage{caption}
\usepackage{subcaption}
\usepackage{mathtools}
\usepackage{enumitem}
\usepackage{xcolor}
\usepackage{tikz}
\usepackage{tcolorbox}

% ---------------------------------------------------------------------------
% Hyperref setup
% ---------------------------------------------------------------------------
\hypersetup{
    colorlinks=true,
    linkcolor=blue,
    citecolor=blue,
    urlcolor=blue,
    pdftitle={Biological Countercurvature of Spacetime},
    pdfauthor={Dr. Sayuj Krishnan S},
}

% ---------------------------------------------------------------------------
% Macros for recurring notation
% ---------------------------------------------------------------------------
% Fields and functions
\newcommand{\Ifield}{I(s)}
\newcommand{\Ifieldt}{I(s,t)}
\newcommand{\gEff}{g_{\mathrm{eff}}}
\newcommand{\Dgeo}{D_{\mathrm{geo}}}
\newcommand{\Dgeohat}{\widehat{D}_{\mathrm{geo}}}
\newcommand{\kappaPassive}{\kappa_{0}}
\newcommand{\kappaInfo}{\kappa_{I}}
\newcommand{\chiK}{\chi_{\kappa}}

% Metric and line element
\newcommand{\dlEff}{d\ell_{\mathrm{eff}}}
\newcommand{\metricEff}{\dlEff^{2} = \gEff(s)\,ds^{2}}

% Scoliosis metrics
\newcommand{\Slat}{S_{\mathrm{lat}}}
\newcommand{\Cobb}{\theta_{\mathrm{Cobb}}}

% Misc
\newcommand{\Reals}{\mathbb{R}}
\newcommand{\dd}{\mathrm{d}}

% ---------------------------------------------------------------------------
% Mathematical Box environment for the countercurvature metric
% ---------------------------------------------------------------------------
\tcbset{
  colback=gray!5!white,
  colframe=black!60,
  fonttitle=\bfseries,
  center title,
}

\newtcolorbox{mathbox}[1][]{
  enhanced,
  breakable,
  title={#1},
}

% ---------------------------------------------------------------------------
% Title / Authors
% ---------------------------------------------------------------------------
\title{Biological Countercurvature of Spacetime: An Information--Cosserat Framework for Spinal Geometry}

\author{Dr.~Sayuj Krishnan S, MBBS, DNB (Neurosurgery)\\
Consultant Neurosurgeon and Spine Surgeon, Yashoda Hospitals, Malakpet, Hyderabad, India\\
\texttt{dr.sayujkrishnan@gmail.com}
}

\date{\today}

% ---------------------------------------------------------------------------
% Document
% ---------------------------------------------------------------------------
\begin{document}
\maketitle

\begin{abstract}
Living systems routinely maintain structure against gravity, from plant stems that grow upward to vertebrate spines that adopt robust S-shaped profiles. We develop a quantitative framework that interprets this behavior as \emph{biological countercurvature}: information-driven modification of the effective geometry experienced by a body in a gravitational field. Drawing on the principles of \emph{Biologic Relativity} and \emph{Active Matter}, we couple an information--elasticity coupling (IEC) model of spinal patterning to three-dimensional Cosserat rod mechanics. This treats the rod in gravity as an analog spacetime and the IEC information field $I(s)$---interpreted here as an \emph{active geometry field}---as a source of effective countercurvature.

Along the body axis $s$, we define a biological metric $d\ell_{\mathrm{eff}}^{2} = g_{\mathrm{eff}}(s)\,ds^{2}$, where the conformal factor $g_{\mathrm{eff}}(s)$ depends on the local amplitude and gradient of $I(s)$. Using this countercurvature metric, we introduce a normalized geodesic curvature deviation $\widehat{D}_{\mathrm{geo}}$ that measures how far information-shaped equilibrium curvature profiles depart from gravity-selected profiles. Across canonical simulations---human-like spinal S-curves, plant-like stems, and microgravity adaptation---$\widehat{D}_{\mathrm{geo}}$ separates gravity-dominated, cooperative, and information-dominated regimes in $(\chi_{\kappa},g)$ space, where $\chi_{\kappa}$ controls information-to-curvature coupling and $g$ denotes gravitational strength.

To probe pathology within the same framework, we introduce a small, localized thoracic asymmetry in the information field or lateral rest curvature and track coronal-plane deformations. In the gravity-dominated regime this perturbation yields negligible lateral deviation. In contrast, in the information-dominated corner of the phase diagram---at stronger coupling than explored in our current parameter sweep---the same small asymmetry is \emph{predicted} to be amplified into a scoliosis-like symmetry-broken branch with increased lateral displacement, Cobb-like angles, and large $\widehat{D}_{\mathrm{geo}}$. Normal sagittal curvature and scoliosis-like patterns thus emerge from a single IEC--Cosserat model operating in different countercurvature regimes, providing an analog-gravity perspective on how biological information reshapes equilibrium geometry in a gravitational field.
\end{abstract}

\section{Significance}

Why does life so often grow and stand ``against'' gravity? We present a mechanical model in which developmental and neural information fields reshape the equilibrium geometry of a spine-like structure in a gravitational field. Using a Cosserat rod in gravity as an analog spacetime and an information--elasticity coupling as a source of ``biological countercurvature,'' we show how normal human-like S-shaped spinal profiles, plant-like upward growth, microgravity adaptation, and scoliosis-like lateral deviations can all emerge within a single framework. A metric derived from the information field lets us treat normal and pathological curvatures as different ``geodesics'' in an information-modified geometry, providing a quantitative language for how biological information can stabilize, redirect, or destabilize shapes that gravity alone would select.

\section{Introduction}

Living systems do not simply obey gravity; they negotiate with it. Human spines adopt robust S-shaped profiles, plant stems grow upward, and neural structures adapt under microgravity. These behaviors suggest that biological information---developmental patterning, neural control, or genetic programs---actively reshapes the equilibrium geometry that gravity alone would select.

We frame this behavior as \emph{biological countercurvature}: information-driven modification of the effective geometry experienced by a body in a gravitational field. This aligns with Torday's principle of \emph{Biologic Relativity}~\cite{torday2016biologic}, which posits that biological function is determined by geometric relationships with the environment. We couple an IEC model to three-dimensional Cosserat rod mechanics, treating the rod in gravity as an analog spacetime and the IEC information field as a source of effective countercurvature. This perspective yields: (1) a biological metric $d\ell_{\mathrm{eff}}^{2} = g_{\mathrm{eff}}(s)\,ds^{2}$ derived from the information field $I(s)$ and its gradient; (2) a normalized geodesic curvature deviation $\widehat{D}_{\mathrm{geo}}$ quantifying how information reshapes equilibrium curvature relative to gravity-selected profiles; and (3) a phase diagram mapping gravity-dominated, cooperative, and information-dominated countercurvature regimes versus information-coupling strength and gravitational loading. Normal sagittal spinal curvature and scoliosis-like lateral deviations then emerge as different regimes of the same model, providing a quantitative link between information processing, mechanics, and geometry in living systems.

% ---------------------------------------------------------------------------
% 2. Methods
% ---------------------------------------------------------------------------
\section{Methods}

\subsection{Information--elasticity coupling and beam model}

On a body axis $s\in[0,L]$, an information field $I(s)$ modulates rest curvature $\kappa_{\mathrm{rest}}(s)$, effective stiffness $E_{\mathrm{eff}}(s)$, and active moments $M_{\mathrm{info}}(s)$ through dimensionless couplings $\chi_{\kappa}$ (curvature), $\chi_{E}$ (stiffness), and $\chi_{M}$ (active moments). The information field $I(s)$ represents spatial patterns of biological activity---developmental gene expression gradients (e.g., HOX/PAX patterning that establishes segmental identity~\cite{wellik2007hox}), neural control signals, or morphogen distributions. In the context of active matter physics~\cite{marchetti2013hydrodynamics}, $I(s)$ acts as an \emph{active geometry field} that drives local deformations against the passive background. A canonical spinal information field peaks in lumbar and cervical regions, corresponding to regions of high neural activity, and $\kappa_{\mathrm{gen}}(s)$ matches typical human sagittal curvature. A static cantilever beam under gravity provides the baseline equilibrium; information-coupled properties modify the solution relative to the passive (gravity-only) case.

\paragraph{Validation.} We validate the boundary-value solver against analytical Euler--Bernoulli sinusoidal solutions under the small-slope regime. With $L=1$\,m and two spatial periods, the numerical centerline matches the analytic profile to machine precision in double-precision arithmetic (L2 error $\lesssim 10^{-3}$ on coarse grids; boundary-condition residuals $\ll 10^{-12}$). Mesh refinement yields the expected decrease in error, confirming consistency of the discretization.

\subsection{Cosserat rod formulation and PyElastica implementation}

We promote the IEC beam to a full three-dimensional Cosserat rod implemented in PyElastica \cite{pyelastica_zenodo,gazzola2018forward,antman2005nonlinear}, accounting for bending, twisting, and stretching with director frames along the rod. The rod is discretized into $n$ elements (typically $n=100$ for full-resolution simulations, $n=50$ for quick sweeps) with element-wise rest curvature, stiffness, and material properties. A clamped base and free end set boundary conditions, with gravity applied as a body force. Information-coupled properties are interpolated to elements. Time integration uses a Position-Verlet scheme with damping (damping coefficient $\gamma \sim 0.1$--$1.0$) to approach static equilibrium. For each experiment, we run the simulation until the maximum velocity falls below a threshold ($<10^{-6}$ m/s), ensuring convergence to static equilibrium.

\subsection{Biological countercurvature metric and geodesic curvature deviation}

On $s\in[0,L]$, we define
\begin{equation}
    d\ell_{\mathrm{eff}}^{2} = g_{\mathrm{eff}}(s)\,ds^{2},
\end{equation}
with
\begin{equation}
    g_{\mathrm{eff}}(s) = \exp\bigl[2\phi(s)\bigr], \qquad
    \phi(s) = \beta_{1}\,\widetilde{I}_{\mathrm{centered}}(s) + \beta_{2}\,\widetilde{I}'(s),
\end{equation}
where $\widetilde{I}\in[0,1]$ is the normalized information field, $\widetilde{I}_{\mathrm{centered}} = \widetilde{I} - \langle\widetilde{I}\rangle$, $\widetilde{I}'$ is the normalized gradient, and $\beta_{1},\beta_{2}>0$ (typically $\beta_{1}=1.0$, $\beta_{2}=0.5$) \cite{lee2018riemannian}. The geodesic curvature deviation between passive and information-coupled curvature profiles is
\begin{equation}
    D_{\mathrm{geo}}^{2}
    = \int_{0}^{L} g_{\mathrm{eff}}(s)\,\bigl[\kappa_{I}(s)-\kappa_{0}(s)\bigr]^{2}\,ds,
\end{equation}
with $\kappa_{0}$ the gravity-only curvature and $\kappa_{I}$ the information-coupled curvature. The normalized form is
\begin{equation}
    \widehat{D}_{\mathrm{geo}}
    = \frac{D_{\mathrm{geo}}}{\sqrt{\int_{0}^{L} g_{\mathrm{eff}}(s)\,\kappa_{0}(s)^{2}\,ds} + \varepsilon},
\end{equation}
with small regularization $\varepsilon$. The metric $g_{\mathrm{eff}}(s)$ weights regions of high information density or gradient; $\widehat{D}_{\mathrm{geo}}$ quantifies departures from gravity-selected geodesics in this countercurvature geometry.

\subsection{Numerical experiments}

\paragraph{Spine-like S-curves.}
A spinal information field $I(s)$ peaks in lumbar and cervical regions; $\kappa_{\mathrm{gen}}(s)$ matches typical human sagittal curvature. We sweep $\chi_{\kappa}\in[0,0.05]$ at $g=9.81\,\mathrm{m/s}^{2}$ with clamped base and free tip.

\paragraph{Plant-like upward growth.}
A passive, sagging rod is compared to an information-driven rod that bends upward. The coupling $\chi_{\kappa}$ controls the transition; $\widehat{D}_{\mathrm{geo}}<0.1$ indicates gravity-dominated sag, while $\widehat{D}_{\mathrm{geo}}>0.2$ indicates strong upward bending.

\paragraph{Microgravity adaptation.}
A fixed information field $I(s)$ is run across $g\in\{1.0, 0.5, 0.1, 0.05, 0.01\}$ (in units of Earth gravity). Passive curvature energy and $\widehat{D}_{\mathrm{geo}}$ are computed to show information-driven structure persisting as gravity is reduced \cite{green2018spinal,marfia2023microgravity}.

\subsection{Thoracic asymmetry and scoliosis metrics}

A localized thoracic bump on normalized $\hat{s}\in[0,1]$ is defined as
\begin{equation}
    G(\hat{s}) = \exp\!\Bigl[-\tfrac{1}{2}\bigl((\hat{s}-\hat{s}_{0})/\sigma\bigr)^{2}\Bigr]
\end{equation}
with $\hat{s}_{0}\approx0.6$ and width spanning $\sim2$--$3$ vertebral levels. An otherwise symmetric field $I_{\mathrm{sym}}$ is perturbed by
\begin{equation}
    I_{\mathrm{asym}}(s) = I_{\mathrm{sym}}(s) + \varepsilon_{\mathrm{asym}}\,\Delta I\,G(\hat{s}),
\end{equation}
with $\Delta I = \max I_{\mathrm{sym}} - \min I_{\mathrm{sym}}$ and $\varepsilon_{\mathrm{asym}}\sim0.05$. Alternatively, a lateral curvature bump $\kappa_{\mathrm{lat}}(s) = \varepsilon_{\mathrm{lat}}G(\hat{s})$ with $\varepsilon_{\mathrm{lat}} = 0.01$--$0.05\,\mathrm{m}^{-1}$ is added to the coronal rest curvature.

From coronal centerlines $(z(s),y(s))$, we define $S_{\mathrm{lat}} = \max_{s}|y(s)|/L_{\mathrm{eff}}$ with $L_{\mathrm{eff}} = \max z - \min z$, and a Cobb-like angle from line fits to the lowest and highest 20\% of points. These metrics are evaluated for symmetric and asymmetric runs across $(\chi_{\kappa},g)$ and combined with $\widehat{D}_{\mathrm{geo}}$ to classify gravity-dominated, cooperative, and scoliosis-like regimes \cite{weinstein2008adolescent,white_panjabi_spine}.

\paragraph{Definition (Biological Countercurvature).}
Let $I(s)$ denote an information field and $\kappa_0(s)$ the baseline curvature under gravity $g$.
We define a target curvature $\kappa_{\mathrm{tgt}}(s)=\kappa_0(s)+\chi_\kappa \partial_s I(s)$ and
an energy weighting $m(s)=1+\alpha I(s)$ that modifies the bending functional
$\mathcal{E}=\int_0^L m(s)\, B\, \kappa(s)^2 \,\mathrm{d}s$.
Countercurvature occurs when the realized curvature opposes the gravitationally induced bend,
i.e., $\int_0^L \kappa(s)\, \kappa_g(s)\,\mathrm{d}s < 0$.
We map $(\alpha,\chi_\kappa)$ producing this inequality to a phase diagram (Fig.~\ref{fig:countercurvature_phase_diagram}).



\section*{Longevity Sub-Study: Ground Sit-to-Stand Frequency}
\textbf{Objective:} Test whether frequent motion against gravity (daily sit-to-stand from the ground) associates with reduced mortality or frailty.

\textbf{Design:} Observational cohort. Primary exposure: daily ground sit-to-stand frequency (questionnaire or wearables). Outcomes: all-cause mortality, frailty index, or gait speed. Confounders: age, sex, BMI, socioeconomic status, smoking, comorbidity, and overall activity level.

\textbf{Analysis:} Cox proportional hazards with robust standard errors; DAG-informed adjustment set; sensitivity to unmeasured confounding (E-value); pre-registered contrasts by tertiles. In the absence of public data here, we include a clearly labeled synthetic demo reproducing the full analysis pipeline, producing Kaplan--Meier curves and adjusted hazard ratios (Fig.~\ref{fig:km_curves}; Table~\ref{tab:cox_results}; ``\textbf{SYNTHETIC---ILLUSTRATIVE}'').



% ---------------------------------------------------------------------------
% 3. Results
% ---------------------------------------------------------------------------
\section{Results}

\subsection{Gravity-selected versus information-selected curvature modes}

In the spine-like configuration, the information field $I(s)$ produces a smooth S-shaped curvature profile $\kappa_{I}(s)$ that is well approximated by a single sign change along the cranio--caudal axis, whereas the purely gravity-selected solution $\kappa_{0}(s)$ tends toward a monotonic C-shaped sag. The stabilized sagittal S-curve is dominated by a single smooth sign-changing mode: $\kappa_{I}(s)$ exhibits only one sign change along the axis and a max-to-RMS curvature ratio of $\approx1.81$, consistent with a sine-like counter-curvature profile against gravity. The normalized geodesic curvature deviation between the gravity-selected and information-selected solutions is $\widehat{D}_{\mathrm{geo}}\approx0.14$, confirming that the information-driven S-curve is not a small perturbation of the passive sag. In this sense, the mature spine behaves as a sinusoidal counter-curvature mode stabilized by IEC.

For the plant-like configuration, varying $\chi_{\kappa}$ drives a transition from passive sag to active upward bending. The geodesic deviation $\widehat{D}_{\mathrm{geo}}$ quantifies this transition, with $\widehat{D}_{\mathrm{geo}}<0.1$ denoting gravity-dominated sag and $\widehat{D}_{\mathrm{geo}}>0.2$ indicating strong information-driven upward bending.

\subsection{Persistence of information-driven shape in microgravity}

Reducing gravity collapses passive curvature energy, yet the information-selected structure persists. As $g$ decreases from $1.0$ to $0.10$ (in Earth units), the normalized geodesic curvature deviation remains essentially unchanged, $\widehat{D}_{\mathrm{geo}}\approx0.091$ (changing by less than 1\% in our simulations), indicating that the information-selected ``spinal wave'' is geometrically stable in microgravity even as the passive response to gravity weakens \cite{green2018spinal,marfia2023microgravity}. This persistence provides quantitative support for the biological countercurvature hypothesis: information fields can maintain structure even when gravitational loading is negligible.

\paragraph{IEC-1: Node drift with preserved wavelength.} Activating $\chi_{\kappa}\partial_{s} I(s)$ shifts the target curvature $\kappa_{\mathrm{tgt}}$ by an information gradient, producing a phase shift $\Delta\phi$ in the centerline while the dominant wavelength remains invariant ($|\Delta\lambda|<2\%$). This reproduces the empirical phenomenology of boundary migration without period change in somitogenesis-like patterns.

\paragraph{IEC-2: Amplitude modulation at near-constant period.} A constitutive gain alters the effective bending response, scaling oscillation amplitude ($A/A_{0}\approx$ prescribed gain) with negligible shift in the spectral peak. Nodes remain stationary to within measurement tolerance.

\subsection{Phase diagram of countercurvature regimes}

We map countercurvature behavior across the $(\chi_{\kappa},g)$ parameter space, where $\chi_{\kappa}$ controls information-to-curvature coupling and $g$ denotes gravitational acceleration. The normalized geodesic deviation $\widehat{D}_{\mathrm{geo}}$ separates distinct regimes: gravity-dominated ($\widehat{D}_{\mathrm{geo}}<0.1$), cooperative ($0.1<\widehat{D}_{\mathrm{geo}}<0.3$), and information-dominated/scoliotic ($\widehat{D}_{\mathrm{geo}}>0.3$). In the present sweep we see gravity-dominated points with $\widehat{D}_{\mathrm{geo}}\approx0.059$ and negligible lateral indices (e.g., $\chi_{\kappa}=0.015$, $g=9.81$) and cooperative points with $\widehat{D}_{\mathrm{geo}}\approx0.15$ and visibly reshaped sagittal curvature (e.g., $\chi_{\kappa}=0.065$, $g=9.81$). Our thresholds for a scoliosis regime ($S_{\mathrm{lat}}\gtrsim0.05$, Cobb-like angles $\gtrsim5^{\circ}$) are not crossed in this parameter window; the symmetry-broken branch remains a predicted extension at larger $\chi_{\kappa}$ or stronger asymmetry rather than a realized regime in the current sweep.

\paragraph{Longevity demo (synthetic).} A preregistered Cox model on a synthetic cohort ($n=10{,}000$) illustrates the analysis path for ground sit-to-stand frequency as an anti-gravity behavior. Kaplan--Meier curves stratified by tertiles and adjusted hazard ratios are reported as a fully reproducible, clearly labeled ``\textbf{SYNTHETIC---ILLUSTRATIVE}'' example pending access to real-world data.

\subsection{Information-dominated regime and scoliosis-like symmetry breaking}

To test left--right asymmetries, we add a small thoracic bump ($\varepsilon_{\mathrm{asym}}\approx5\%$) to $I(s)$ or the lateral rest curvature. Symmetric ($\varepsilon_{\mathrm{asym}}=0$) and asymmetric runs are simulated over $(\chi_{\kappa},g)$. From coronal projections we compute $S_{\mathrm{lat}}$ and Cobb-like angles, alongside $\widehat{D}_{\mathrm{geo}}$.

In gravity-dominated regions (low $\chi_{\kappa}$, high $g$), symmetric and asymmetric solutions are nearly identical: $S_{\mathrm{lat}}$ and Cobb-like angles change by only a few percent or degrees, and $\widehat{D}_{\mathrm{geo}}$ remains small. The perturbation is effectively suppressed by gravity-selected curvature. As we move toward the information-dominated corner of parameter space (high $\chi_{\kappa}$ at moderate or reduced $g$), the model predicts that the same perturbation can produce pronounced lateral deformation: $S_{\mathrm{lat}}\gtrsim0.05$, Cobb-like angles $\gtrsim5$--$10^{\circ}$, and $\widehat{D}_{\mathrm{geo}}$ in the large-deviation regime. In this regime, the information field reshapes the effective metric so strongly that a small asymmetry is expected to be amplified into a scoliosis-like branch. Thus scoliosis-like patterns can arise when countercurvature dominates gravity, without invoking a fundamentally different mechanical mechanism \cite{weinstein2008adolescent,white_panjabi_spine}.

% ---------------------------------------------------------------------------
% 4. Discussion
% ---------------------------------------------------------------------------
\section{Discussion}

\subsection{Countercurvature regimes}

The phase diagram quantifies how biological information and gravity interact. In gravity-dominated regimes, the rod follows gravity-selected geodesics; information plays little role. In cooperative regimes, information reshapes curvature within the gravitational background. In information-dominated regimes, countercurvature governs the geometry and enables symmetry breaking. The normalized geodesic curvature deviation $\widehat{D}_{\mathrm{geo}}$ provides a quantitative measure of these interactions and transitions.

\subsection{Growth against gravity as a standing mode}

The adult sagittal spine can be interpreted as a standing counter-curvature mode selected by an information field acting against gravity, not as a passive beam sagging under load. As coupling increases, the system transitions from a C-shaped profile to a robust, sine-like S-curve that persists when gravity is reduced. This view extends to development (progressive recruitment of higher curvature modes) and to pathology, where the same machinery can amplify small asymmetries into lateral branches when information dominates.

\subsection{Analog gravity interpretation}

``Countercurvature of spacetime'' here is analog rather than fundamental: the Cosserat rod in a uniform gravitational field is the effective spacetime, and $I(s)$ modifies $d\ell_{\mathrm{eff}}^{2}=g_{\mathrm{eff}}(s)\,ds^{2}$. The quantity $\widehat{D}_{\mathrm{geo}}$ measures how strongly information reshapes equilibrium geometry relative to gravity-selected solutions, in analogy with additional fields modifying geodesics in general relativity \cite{einstein1916grundlage,wald1984gr}. We do not propose any modification of Einstein's equations; the analog language organizes how developmental and neuromuscular information can select, stabilize, or destabilize curvature modes of the spine in a gravitational background. This is consistent with the view of the spine as \emph{active matter} on a curved manifold~\cite{marchetti2013hydrodynamics,goriely2017mathematics}, where the ``activity'' is the information-driven deviation from the passive geodesic.

\subsection{Thermodynamics of Counter-curvature}

The competition between gravity and information can be framed thermodynamically. Gravity tends to increase the entropy of the rod configuration (buckling, collapse to a lower energy state), while the information field $I(s)$ acts to reduce entropy by maintaining a specific, ordered target shape (the S-curve). The stable spinal configuration minimizes a generalized action $\mathcal{S} = \mathcal{S}_{\mathrm{grav}} + \mathcal{S}_{\mathrm{info}}$, where $\mathcal{S}_{\mathrm{info}}$ represents the information-theoretic cost of maintaining the counter-curvature against the gravitational tendency toward disorder. This aligns with recent proposals applying Information Field Theory to biological self-organization~\cite{ensslin2009information}.

\subsection{Implications for scoliosis and control}

Scoliosis-like patterns arise when countercurvature dominates: small asymmetries are amplified into lateral deviations. This quantifies how developmental or neuromuscular asymmetries might yield pathological curvature. The phase diagram suggests such behavior when information--curvature coupling is strong and gravity is moderate or reduced. Normal sagittal curvature and scoliosis-like deviations then appear as regimes of the same IEC mechanism, suggesting that interventions could target the coupling itself rather than treat them as separate phenomena.

Ciliary flow patterns provide a concrete biological example of information fields that can break left--right symmetry: coordinated ependymal cell cilia beating generates cerebrospinal fluid (CSF) flow gradients that establish spatial information fields~\cite{grimes2016zebrafish}. Disruptions in ciliary function lead to abnormal CSF flow patterns and are associated with elevated scoliosis incidence, consistent with the IEC framework's prediction that asymmetric information gradients can amplify into pathological curvature in the information-dominated regime.

% ---------------------------------------------------------------------------
% 5. Limitations and Outlook
% ---------------------------------------------------------------------------
\section{Limitations and Outlook}

The information field $I(s)$ is phenomenological; its form and couplings are chosen to match observed curvature. The countercurvature metric $g_{\mathrm{eff}}(s)$ is heuristic, with empirical weights $\beta_{1},\beta_{2}$. Most experiments use a simplified beam; full 3D Cosserat models are applied mainly to the scoliosis analysis and should be extended. The normalized geodesic curvature deviation $\widehat{D}_{\mathrm{geo}}$ can inflate as $g\to0$ because the passive energy denominator collapses; this is expected but merits care in the microgravity limit. In 2D beam models, $S_{\mathrm{lat}}$ and Cobb-like angles use a pseudo-coronal projection; full 3D rods provide the true coronal plane.

Future work includes: (1) applying the framework to experimental microgravity and clinical scoliosis data; (2) deriving $I(s)$ from developmental or neural control principles (e.g., HOX/PAX patterning and cilia-driven flows); (3) relating information--curvature coupling to known biological processes in spinal development and control; and (4) exploring therapies that target the coupling itself.

% ---------------------------------------------------------------------------
% 6. Conclusion
% ---------------------------------------------------------------------------
\section{Conclusion}

We present a quantitative framework for biological countercurvature that unifies normal sagittal curvature and scoliosis-like deviations within a single IEC model. A biological metric $d\ell_{\mathrm{eff}}^{2}=g_{\mathrm{eff}}(s)\,ds^{2}$, a normalized geodesic curvature deviation $\widehat{D}_{\mathrm{geo}}$, and a phase diagram of countercurvature regimes together show that information-driven structure maintenance persists in microgravity and that normal and pathological patterns are regimes of the same model. The analog-gravity perspective---treating a rod in gravity as an effective spacetime and information fields as sources of countercurvature---links information processing, mechanics, and geometry in living systems. Extending this framework to data, first-principles information fields, and therapeutic strategies targeting information--curvature coupling are natural next steps.

% ---------------------------------------------------------------------------
% Code and Data Availability
% ---------------------------------------------------------------------------
\section*{Code Availability}

All simulations and analyses used the open-source Python package \texttt{spinalmodes} (v0.1.0), available at \url{https://github.com/sayujks0071/life}. The package implements the IEC beam solver, countercurvature metric, geodesic curvature deviation, scoliosis metrics, and PyElastica-based Cosserat rod simulations. Key functions include \texttt{compute\_countercurvature\_metric}, \texttt{geodesic\_curvature\_deviation}, \texttt{compute\_scoliosis\_metrics}, \texttt{classify\_scoliotic\_regime}, and \texttt{solve\_beam\_static}. Experiment scripts (microgravity, spine modes, plant growth, phase diagram, scoliosis) are in \texttt{src/spinalmodes/experiments/countercurvature/} with documented CLIs and shell helpers. Minimal examples are in \texttt{examples/quickstart.py} and \texttt{examples/quickstart.ipynb}. The exact version used here is archived as release v0.1.0 (see \texttt{CITATION.cff}).

\section*{Data Availability}

All data underlying the figures are stored as CSV or are reproducible from the provided code. Experiment outputs (curvature profiles, countercurvature metrics, geodesic deviations, scoliosis indices) are written to \texttt{outputs/} by the scripts in \texttt{src/spinalmodes/experiments/countercurvature/}. For each figure panel, mappings from script to output files are listed in \texttt{docs/manuscript\_code\_data\_availability.md}. Running the experiments with default parameters (or \texttt{--quick} for reduced resolution) reproduces all CSVs used by the figure-generation scripts. No proprietary or patient-identifiable data are used.

\section*{Figures}

\begin{figure}[h!]
    \centering
    \begin{subfigure}[b]{0.48\textwidth}
        \centering
        \includegraphics[width=\textwidth]{fig_countercurvature_panelA.pdf}
        \caption{Curvature profiles $\kappa_{0}(s)$ vs $\kappa_{I}(s)$.}
    \end{subfigure}
    \hfill
    \begin{subfigure}[b]{0.48\textwidth}
        \centering
        \includegraphics[width=\textwidth]{fig_countercurvature_panelB.pdf}
        \caption{Countercurvature metric $g_{\mathrm{eff}}(s)$ and information field.}
    \end{subfigure}

    \vspace{0.5em}

    \begin{subfigure}[b]{0.48\textwidth}
        \centering
        \includegraphics[width=\textwidth]{fig_countercurvature_panelC.pdf}
        \caption{Normalized geodesic deviation $\widehat{D}_{\mathrm{geo}}$ vs $\chi_{\kappa}$.}
    \end{subfigure}
    \hfill
    \begin{subfigure}[b]{0.48\textwidth}
        \centering
        \includegraphics[width=\textwidth]{fig_countercurvature_panelD.pdf}
        \caption{Microgravity adaptation: $\widehat{D}_{\mathrm{geo}}$ vs $g$.}
    \end{subfigure}

    \caption{Information-driven countercurvature in spine-like and plant-like configurations. 
    Panel A shows the curvature profiles for passive (gravity-only) and information-coupled configurations. Panel B displays the countercurvature metric $g_{\mathrm{eff}}(s)$ along the rod axis, highlighting regions of high information processing. Panel C demonstrates how normalized geodesic deviation $\widehat{D}_{\mathrm{geo}}$ increases with information--curvature coupling strength $\chi_{\kappa}$. Panel D shows that $\widehat{D}_{\mathrm{geo}}$ persists as gravitational loading is reduced, while passive curvature energy collapses, demonstrating information-driven structure maintenance in microgravity.
    }
    \label{fig:countercurvature_main}
\end{figure}

\begin{figure}[h!]
    \centering
    \includegraphics[width=0.65\textwidth]{fig_phase_diagram_scoliosis.pdf}
    \caption{Phase diagram in $(\chi_{\kappa},g)$ showing gravity-dominated, cooperative, and information-dominated regimes. Contours: $\widehat{D}_{\mathrm{geo}}$. Markers: points where a small thoracic asymmetry ($\varepsilon_{\mathrm{asym}}=0.05$) produces a scoliosis-like branch (high $S_{\mathrm{lat}}$ and Cobb-like angles). The scoliotic regime (shaded) emerges in the information-dominated corner where $\widehat{D}_{\mathrm{geo}}>0.3$ and small asymmetries are predicted to be amplified.}
    \label{fig:phase_diagram}
\end{figure}

% ---------------------------------------------------------------------------
% References
% ---------------------------------------------------------------------------
\bibliographystyle{unsrtnat}
\bibliography{refs}

% Note: The bibliography file `refs.bib` contains a minimal but coherent
% seed covering PyElastica, Cosserat rods, Riemannian geometry, GR,
% scoliosis, and microgravity. Extend as needed with additional references.

\end{document}
